%%%\documentclass[handout]{beamer}
\documentclass[handout]{beamer}
\usepackage{pdfpages}
\newcommand{\comment}[1]{}
\usetheme{Copenhagen}
\usecolortheme{beaver}
%%% Intro, basic documents, Maths mode, figures/tables, cross-referencing
%%%session 1: Intro and basic documents: resources, philosophy, steps to Latex, document style, margins, syntax \ {} , accents, dashes, enumerate, paragraphs
%%%session 2: Maths mode: dollars/displayed eq (mention font), basic commands: greek letters, \frac, \sin, \cos, subscript, spacing, equation environments, theorem environments, adjustable brackets, arrays, newcommands
%%%session 3: cross-referencing: labels, bibliography
%%%session 4: tables & figures, 

\renewenvironment{proof}{{\scshape Proof. }\itshape }{\hfill$\spadesuit$\par}


%\renewtheorem{theorem}{Theorem}[section]
\newtheorem{conjecture}[theorem]{Conjecture}

\newcommand{\myfrac}[2]{\frac{x^2+#1}{#2}}
\newcommand{\myint}{\int_0^\infty}


\title{An introduction to \LaTeX}
\author[Emma McCoy]{Emma McCoy\\Course Notes: Emma McCoy/Phillip Kent}

\begin{document}
					
\begin{frame}
\maketitle
\end{frame}

\section{Introduction}
\begin{frame}[fragile]{Resources}
This course is for \LaTeX\  ``$2_\varepsilon$'' (current standard).\\
The previous version ``2.09'' works slightly differently - beware!\\[0.5cm]
\textbf{Course website}:
\verb|http://www.ma.ic.ac.uk/~ejm/LaTeX/Website/|\\[0.5cm]
\textbf{References}: 
\begin{itemize}
\item {\em \LaTeX\ User Guide \&
Reference Manual} by Leslie Lamport (Second edition, Addison-Wesley,  1994).
\item {\em A Guide to \LaTeX}  by Kopka \& Daly (Third edition,
Addison-Wesley, 1999) (details more advanced features).
\end{itemize}
\end{frame}

\begin{frame}{Course contents}
\begin{enumerate}
\item Introduction \pause
\item A first \LaTeX\ document \pause
\item Maths mode \pause
\item Complex documents \pause
\item Figures, tables, etc...
\end{enumerate}
\end{frame}

\begin{frame}{Pros}
\begin{itemize}
\item Great for typing mathematics
\item Great for complex documents  -- cross-referencing, labelling, bibliographies...
\item \LaTeX\ output is beautiful -- virtually of professional-typeset quality
\item The basic \LaTeX\ system is FREE.
\item \LaTeX\ makes typing ``easy'':
\begin{itemize}
\item formatting is automatic 
\item emphasis on content over formatting
\end{itemize}
\item \LaTeX\ is written as plain text 
\begin{itemize}
\item compact, portable
\item transferable across the internet/email
\end{itemize}
\item accepted by all major academic publishers 
\begin{itemize}
\item speeds up the publishing process
\item reduces the chance of printing errors
\end{itemize}
\end{itemize}
\end{frame}

\begin{frame}{Cons}
When is it not appropriate?\\[0.5cm] Many \LaTeX\ users make their CVs, presentations and even address their letters in \LaTeX\ but it's not great for:
\begin{itemize}
\item documents with little text and lots of pictures.
\item incorporating spreadsheets etc. into text.
\end{itemize}
\end{frame}


\begin{frame}{Where?}
\begin{itemize}
\item Free versions of \LaTeX\ exist across all platforms -- Microsoft Windows, Linux, Unix and Apple Macintosh. 
\item Commercial (non-free) versions offer some extra features, e.g. WYSIWYG package \emph{Scientific Word} or {\em BaKoMa TeX Word}. 
\item Power users should use Emacs.
\item We will use MiK\TeX: {\tt http://miktex.org/}
\end{itemize}
\end{frame}

\begin{frame}{About \TeX}
\begin{itemize}
\item \LaTeX\ is a super-set (macro package) of the typesetting language \TeX\, created by Donald Knuth
\item Plain \TeX\ needs programming skills  -- deliberate policy
\item \LaTeX\ (originally created by Leslie Lamport) adds functionality
\item Originally other \TeX\ macro packages -- e.g. $\cal AMS$\TeX\ and $\cal AMS$\LaTeX\ -- now, incorporated into \LaTeX\
\item \LaTeX\ and \TeX\ are not two different languages
\begin{itemize}
\item Most \TeX\ commands work in \LaTeX
\item Sometimes a conflict where \LaTeX\ has re-defined a \TeX\ command
\end{itemize}
\item The {\em \TeX book} by Donald Knuth -- standard reference for \TeX
\end{itemize}
\end{frame}

\begin{frame}[fragile]{Writing \LaTeX}
\begin{itemize}
\item Writing \LaTeX\ is like writing computer programs in, say, Fortran or C -- and there will be similar frustrations
\item  \LaTeX\ {\emph source file}: contains plain (ASCII) text and formatting commands
\item Commands are preceded by a ``{\tt \textbackslash}''.
\item Nine \emph{reserved} characters:
\begin{verbatim}
 \ % $ ^ _ & # ~ { }
\end{verbatim}
If you want a ``\%'', type: \verb|\%|
\item Source file must end in ``\verb|.tex|''
\end{itemize}
\end{frame}

\begin{frame}{The steps of writing in \LaTeX}
\begin{figure}[h]
\begin{center}
\setlength{\unitlength}{0.035cm}
\begin{picture}(210,140)(30,0)
\put(5,70){\framebox(60,15){EDIT}}
\put(90,115){\framebox(60,15){COMPILE}}
\put(90,20){\framebox(60,15){PREVIEW}}
\put(170,70){\framebox(60,15){PRINT}}
\put(90,135){\dashbox{5}(60,15){\tt\small TeX|(TeX)}}
\put(90,0){\dashbox{5}(60,15){\tt\small View|(View)}}
\put(235,70){\dashbox{5}(60,15){\tt\small View+Print}}
\put(120,110){\vector(0,-1){70}}
\put(35,90){\vector(4,3){45}}
\put(80,30){\vector(-4,3){45}}
\put(155,126){\vector(4,-3){45}}
\put(155,28){\vector(4,3){45}}
\end{picture}
\end{center}
%\caption{The steps of doing \LaTeX.}
\label{fig.steps}
\end{figure}
\end{frame}

\section{A first \LaTeX\  document}
\begin{frame}[fragile]{A skeleton source file}
\hphantom{something}
{\footnotesize
\begin{verbatim}
% the essential components of a LaTeX file
% (N.B. % is the comment character, everything to 
% the right of it on a line is IGNORED.)

\documentclass{article}
% **** PREAMBLE **** 
% title/author/date information
% definitions, short-hands, macros etc. BUT NO text

\begin{document}
% **** BODY OF DOCUMENT ****
% ...the text itself
% N.B. the RESERVED CHARACTERS:
%     \ % $ ^ _ & # ~ { }

\end{document}
\end{verbatim}
}
\end{frame}

\subsection{Commands that affect the whole document}
\begin{frame}[fragile]{Document class}
\begin{verbatim}
\documentclass[options]{style}
\end{verbatim}
\begin{tabbing}
\hspace{1cm} \= style: \hspace{1cm} \= \verb+book+ \\
\>\> \verb+report+ \\
\>\> \verb+article+ \\
\>\> \verb+letter+ \\
\> options: \> \verb+11pt+ \\
\>\> \verb+12pt+ \\
\>\> \verb+a4paper+
\end{tabbing}
\end{frame}

\begin{frame}[fragile]{Sectioning Commands}
\begin{itemize}
\item \verb+\chapter+ 
\item \verb+\section+ 
\item \verb+\subsection+ 
\item \verb+\subsubsection+ 
\end{itemize}
Example: \verb+\chapter{title}+
\end{frame}

\begin{frame}[fragile]{Margin sizes}
If you are not happy with margin sizes they can be adjusted:
\begin{itemize}
\item \verb+\setlength{\textwidth}{5.7cm}+
\item \verb+\setlength{\oddsidemargin}{0.6in}+
\item \verb+\setlength{\topmargin}{-0.5in}+
\item \verb+\setlength{\textheight}{246mm}+
\end{itemize}
alternatively use:
\begin{verbatim}
      \addtolength{\topmargin}{-5mm}
\end{verbatim}
\end{frame}



\subsection{Writing in \LaTeX}
\begin{frame}[fragile]{Font size}
\begin{itemize}
\item \verb+\small+ 
\item \verb+\normalsize + 
\item \verb+\large+  \verb+\Large+  \verb+\LARGE+
\verb+\huge+ \verb+\Huge+ 
\end{itemize}
Example: \verb+{\Large this will be large}+
\end{frame}

\begin{frame}[fragile]{Font style}
\begin{itemize}
\item Bold: \verb+\bf+ \hspace{.5cm} 
\begin{itemize}
\item Example: \verb+{\bf this will be bold}+ 
\end{itemize}
\item Italics: \verb+\it+ 
\begin{itemize}
\item There are usually several command to achieve  the same result: 
\verb+{\em italic} \textit{italic} \emph{italic} {\it italic}+ \\
will all produce: \emph{italic}
\end{itemize}
\end{itemize}
\end{frame}

\begin{frame}[fragile]{Numbered list}
For a numbered list:
{\footnotesize
\begin{verbatim}
\begin{enumerate}
\item This is the first item
\item here's the second
\begin{enumerate}
\item this will be part 1 of number 2
\item this is part 2 
\end{enumerate}
\end{enumerate}
\end{verbatim}

{\em Output of above commands:}
\begin{enumerate}
\item This is the first item
\item here's the second
\begin{enumerate}
\item this will be part 1 of number 2
\item this is part 2
\end{enumerate}
\end{enumerate}
Replace \verb+enumerate+ with \verb+itemise+ for bullet points
}
\end{frame}

\begin{frame}[fragile]{Extra critical commands}
\begin{itemize}
\item to go to a new page use: \verb+ \newpage+ 
\item to go to a new line use: \verb+ \newline+ or \verb+\\+ 
\item to start a new paragraph: leave a blank line
\item to prevent indenting use: \verb+ \noindent+ 
\item For double spacing, in the preamble:
\begin{verbatim}
\renewcommand{\baselinestretch}{1.6}
\end{verbatim}
\end{itemize}
\end{frame}

\begin{frame}[fragile]{Other useful commands}
\begin{itemize}
\item Quotation marks: use \verb+``a''+ to produce ``a''
\item Accents: use \verb+\'e, + \verb+\"e ,+ \verb+\^e+ to produce \'e, \"e, \^e.
\item Dashes: use \verb+--+, \verb+---+ to produce -- and ---
\end{itemize}
\end{frame}

\begin{frame}[fragile]{Preliminary Exercise}
\begin{itemize}
\item Open \TeX works
\item Follow the instructions on the sheet
\end{itemize}
\end{frame}

\begin{frame}[fragile]{Exercise 1}
\begin{verbatim}
http://www2.imperial.ac.uk/~ejm/
       LaTeX/Website/exercises/exercise1.html
\end{verbatim}
\end{frame}


\section{Maths mode}
%%%session 2: Maths mode: dollars/displayed eq (mention font), basic commands: greek letters, \frac, \sin, \cos, subscript, spacing, equation environments, theorem environments, adjustable brackets, arrays, newcommands
\begin{frame}[fragile]{Math mode}
Maths is ``expensive'':
\begin{tabbing}
\verb+\[+ \hspace{.5cm} \= \verb+\]+ 
\hspace{.2cm} \= or \hspace{.5cm} \= \verb+$$+ 
\hspace{.5cm} \= \verb+$$+ \= -- displayed formula \\
\verb+\(+ \> \verb+\)+ \> or \> \verb+$+ \> \verb+$+  \> -- in-text formula 
\end{tabbing}
\textbf{E.g.}
\begin{verbatim}
I could put $x = y+2z+3w$ in the text 
or as a displayed equation:
\[x = y+2z+3w\]
\end{verbatim}
\textbf{gives:} \\
I could put $x = y+2z+3w$ in the text or as a displayed equation:
\[x = y+2z+3w\]
\end{frame}

\subsection{Basic typesetting of maths}
\begin{frame}[fragile]{Subscript/superscripts:}
\begin{tabbing}
$x^{2y}$ \hspace{.5cm} \= \verb+$x^{2y}$+ \\
$x_1^{y^{2}}$ \> \verb+$x_1^{y^{2}}$+
\end{tabbing}
Note bracketing, more than one argument in the sub/superscript must be enclosed
in $\{ \ldots \}$.
\begin{verbatim}
\[
\int_0^\infty f(t) \, dt
\]
\end{verbatim}
{\em Output of above:}\\ 
\[
\int_0^\infty f(t) \, dt
\]
\end{frame}

\begin{frame}[fragile]{Greek letters}
Remember your Greek letters:
\verb+$\alpha$+, \verb+$\beta$+, \verb+$\gamma$+, \verb+$\kappa$+, produce:\\
$\alpha$, $\beta$, $\gamma$, $\kappa$.\\[0.5cm]
...and just capitalise to get (non-Arabic) capital Greek letters, e.g. \verb+$\Gamma$+ produces $\Gamma$.


\end{frame}

\begin{frame}[fragile]{Numbered equations}
\begin{verbatim}
\begin{equation}
S_2 = \sum_{i=1}^N x_i^2 + 
\sum_{i=1}^N (y_i-\overline{y})^2 
\end{equation}
\end{verbatim}
{\em Output of above commands:}\\ 
\begin{equation}
S_2 = \sum_{i=1}^N x_i^2 + \sum_{i=1}^N (y_i-\overline{y})^2 
\end{equation}
\end{frame}

\begin{frame}[fragile]{Fractions}
\[
x= \frac{y+z/2}{y^2+1} 
\]
\begin{verbatim}
\[
x= \frac{y+z/2}{y^2+1} 
\]
\end{verbatim}
It's considered bad practice to \verb+\frac+ in in-text formulas because it basically looks ugly: $x= \frac{y+z/2}{y^2+1}$.
\end{frame}

\begin{frame}[fragile]{Adjustable brackets}
Use \verb+\left+ and \verb+\right+ for correct sizing
\[
\left\{ \left[\frac{1}{2}\right] - \left[\frac{1}{4}\right] \right\}
\]
\begin{verbatim}
\[
\left\{ \left[\frac{1}{2}\right] - 
\left[\frac{1}{4}\right] \right\}
\]
\end{verbatim}
You can use \verb+\left\{+, \verb+\left[+, \verb+\left(+, \verb+\left/+, \verb+\left.+ etc...
\end{frame}

\begin{frame}[fragile]{Spacing}
The \verb+\quad+ command leaves some space, other spaces in maths mode can
be created with the following commands (smallest first): \\
\verb+ \, \; \quad \qquad+

\[
x \, x \; x \quad x \qquad x
\]
is produced by:
\begin{verbatim}
\[
x \, x \; x \quad x \qquad x
\]
\end{verbatim}

\end{frame}

\begin{frame}[fragile]{Arrays}
\[
x= \left\{
\begin{array}{cl}
y  \qquad  & \mbox{if $y>0$} \\
z+y \qquad & \mbox{otherwise}
\end{array}
\right.
\]
\begin{verbatim}
\[
x= \left\{
\begin{array}{cl}
y \quad & \mbox{if $y>0$} \\
z+y \quad & \mbox{otherwise}
\end{array}
\right.
\]
\end{verbatim}
\end{frame}

\begin{frame}[fragile]{Lining up in columns}
To produce:
\begin{align*}
\rho_t + (\rho u)_x + (\rho v)_y &= 0,\\
u_t + uu_x + vu_y + \frac{1}{\rho} p_x &= 0, \\
v_t + uv_x + vv_y + \frac{1}{\rho} p_y &= 0.
\end{align*}
\begin{verbatim}
\begin{align*}
\rho_t + (\rho u)_x + (\rho v)_y &= 0,\\
u_t + uu_x + vu_y + \frac{1}{\rho} p_x &= 0,\\
v_t + uv_x + vv_y + \frac{1}{\rho} p_y &= 0.
\end{align*}
\end{verbatim}
Use the \& symbol to line up the columns.
\end{frame}

\begin{frame}[fragile]{Numbered lines}
Use \verb+\begin{align}+ for numbered equations -- you can suppress numbering for an individual equation by using the \verb+\nonumber+ command before \verb+\\+.
\begin{align}
\rho_t + (\rho u)_x + (\rho v)_y &= 0,\\
u_t + uu_x + vu_y + \frac{1}{\rho} p_x &= 0,\nonumber\\
v_t + uv_x + vv_y + \frac{1}{\rho} p_y &= 0.
\end{align}
\begin{verbatim}
\begin{align}
\rho_t + (\rho u)_x + (\rho v)_y &= 0,\\
u_t + uu_x + vu_y + \frac{1}{\rho} p_x &= 0,\nonumber\\
v_t + uv_x + vv_y + \frac{1}{\rho} p_y &= 0. 
\end{align}
\end{verbatim}
\end{frame}

\begin{frame}[fragile]{Matrices}
\[P = \left( \begin{array}{ccc} 
1 & \cdots & 3\\
\vdots &\ddots & \vdots\\
1 & \cdots & 3
\end{array}\right)\]
is produced by:
\begin{verbatim}
\[P = \left( \begin{array}{ccc} 
1 & \cdots & 3\\
\vdots &\ddots & \vdots\\
1 & \cdots & 3
\end{array}\right)\]
\end{verbatim}
\end{frame}


\subsection{New commands}
\begin{frame}[fragile]{Commands/Functions:}
Often you will find yourself repeating the same commands to produce complicated constructions, e.g. you might find yourself repeatedly typesetting \verb+\int_0^\infty+ to produce 
\[\int_0^\infty\]
Save yourself time with \verb+\newcommand+ in the {\em preamble}:
\begin{verbatim}
\newcommand{\myint}{\int_0^\infty}
\end{verbatim}
Then in the document type (for example) \verb+\myint x \, dx+ to obtain:
\[\myint x \, dx\]
\end{frame}

\begin{frame}[fragile]{Multiple arguments}
You can give arguments to \verb+\newcommand+: \\
E.g. if we want to type:
\[
\frac{x^2+a}{b}
\]
where the values of $a$ and $b$ can change,
\begin{verbatim}
\newcommand{\myfrac}[2]{\frac{x^2+#1}{#2}}
\end{verbatim}
Then use
\begin{verbatim}
\[
y=\myfrac{2}{4}
\]
\end{verbatim}
to produce
\[
y=\myfrac{2}{4}
\]
\end{frame}
\begin{frame}[fragile]{New environments}
In the preamble:
\begin{verbatim}
\newenvironment{proof}{{\scshape Proof. }\itshape }
{\hfill$\spadesuit$\par}
\end{verbatim}
Then in body:
\begin{verbatim}
\begin{proof}
Let us start by considering whether there is 
actually anything to prove. Turns out there isn't.
\end{proof}
\end{verbatim}
gives:\\
\begin{proof}
Let us start by considering whether there is actually anything to prove. Turns out there isn't.
\end{proof}
\end{frame}
\begin{frame}[fragile]{Theorems}
In preamble:
\begin{verbatim}
\newtheorem{theorem}{Theorem}[section]
\newtheorem{conj}[theorem]{Conjecture}
\end{verbatim}
Then in body:
\begin{verbatim}
\begin{theorem}[Something] something \end{theorem}
\begin{conj}[Something else] something else \end{conj}
\end{verbatim}
gives:\\
\begin{theorem}[Something] something \end{theorem}
\begin{conjecture}[Something else] something else \end{conjecture}
\end{frame}

\begin{frame}[fragile]{Exercise 2}
\begin{verbatim}
http://www2.imperial.ac.uk/~ejm/
       LaTeX/Website/exercises/exercise2.html
\end{verbatim}
\end{frame}

\section{Tables and Figures}
\begin{frame}[fragile]{Tables:}
To produce the following table:
\renewcommand{\arraystretch}{1.4}
\begin{table}[h]
\begin{center}
\begin{tabular}{|l|c|c|} \hline
& \multicolumn{2}{|c|}{Statistic} \\ \hline
Distribution & Expected value & Variance \\ \hline\hline
Binomial($n,p$) & $np$ & $np(1-p)$ \\ \hline
Uniform($\alpha, \beta$) & $(\beta+\alpha)/2$ & $(\beta-\alpha)^2/12$ \\ \hline
Exponential($\lambda$) & $1/\lambda$ &  $1/\lambda^2$ \\ \hline
\end{tabular}
\end{center}
\caption{Means and variances}
\end{table}
\end{frame}

\begin{frame}[fragile]{Table code}
The code: 
{\footnotesize
\begin{verbatim}
\renewcommand{\arraystretch}{1.4}
\begin{table}[h]
\begin{center}
\begin{tabular}{|l|c|c|} \hline
& \multicolumn{2}{|c|}{Statistic} \\ \hline
Distribution & Expected value & Variance \\ \hline\hline
Binomial($n,p$) & $np$ & $np(1-p)$ \\ \hline
Uniform($\alpha, \beta$) & $(\beta+\alpha)/2$ & 
     $(\beta-\alpha)^2/12$ \\ \hline
Exponential($\lambda$) & $1/\lambda$ &  
     $1/\lambda^2$ \\ \hline
\end{tabular}
\end{center}
\caption{Means and variances}
\end{table}
\end{verbatim}
}
\end{frame}

\begin{frame}[fragile]{Extra useful table commands:}
Can have a fixed width box as one of the columns (to allow line breaks):
\begin{verbatim}
\begin{tabular}{|l|p{5cm}|} \hline
First & extremely clear and accurate 
description of the school, the role in the
classroom and the teaching methods used \\ \hline
Upper Second & clear and accurate description 
of the school, the role in the
classroom and the teaching methods used \\ \hline
Lower Second & a description of the school, 
the role in the classroom and the
teaching methods used \\ \hline
\end{tabular}
\end{verbatim}
\end{frame}


\begin{frame}{The table}
\begin{tabular}{|l|p{8cm}|} \hline
First & extremely clear and accurate 
description of the school, the role in the
classroom and the teaching methods used \\ \hline
Upper Second & clear and accurate description 
of the school, the role in the
classroom and the teaching methods used \\ \hline
Lower Second & a description of the school, 
the role in the classroom and the
teaching methods used \\ \hline
\end{tabular}
\end{frame}


\begin{frame}[fragile]{Multirow}
There is also a \verb+\multirow+ command, but you need to 
add \verb+\usepackage{multirow}+\\
usage: \verb+\multirow{number of rows to span}{alignment}+ \\
can set alignment to \verb+*+ for best fit.

Similarly, \verb+\usepackage{multicolumn}+.
\end{frame}

\begin{frame}[fragile]{Aligning to decimal point}
\begin{verbatim}
\begin{tabular}{r@{.}l}
  2&1\\
  16&2\\
  2&456\\
\end{tabular}
\end{verbatim}
gives:
\begin{tabular}{|r@{.}l|}
\hline
  2&1\\
  16&2\\
  2&456\\
\hline
\end{tabular}
\end{frame}

\begin{frame}{Figures}
To produce the following picture from a PDF file:
\begin{figure}[h]
\begin{center}
\includegraphics[height=3cm,width=4cm]{sh05plume.jpg}
\caption{Random figure}
\end{center}
\end{figure}
\end{frame}

\begin{frame}[fragile]{Code}
In the preamble use the \verb+graphicx+ package:
\begin{verbatim}
\usepackage{graphicx}
\end{verbatim}
Then use the following commands:
\begin{verbatim}
\begin{figure}[h]
\begin{center}
\includegraphics[height=4cm,width=6cm]{Rplots.pdf}
\caption{Random figure}
\end{center}
\end{figure}
\end{verbatim}
\end{frame}

\begin{frame}[fragile]{Rotating figures and tables}
To rotate figures and tables use the \verb+rotating+ package:
include the following line in the preamble:
\begin{verbatim}
\usepackage{rotating}
\end{verbatim}
Then use \verb+\begin{sidewaysfigure}+ or \verb+\begin{sidewaystable}+.
\end{frame}

\begin{frame}[fragile]{Exercise}
\begin{enumerate}
\item Pick a random picture from the web and put it in your document.
\item Tables: \\
\begin{verbatim}
http://www2.imperial.ac.uk/~ejm/LaTeX/
           Website/exercises/table.html
\end{verbatim}
\end{enumerate}
\end{frame}

\section{Writing complex documents}
%the start of a document: title, author, date, list of tables, list of figures
%labels and referring
%bibtex
%messing around with the numbers.

\subsection{At the start of the document}
\begin{frame}[fragile]{Title}
In the preamble type: \\
\begin{verbatim}
\title{A snappy title}
\author{Emma McCoy}
\date{\today}
\end{verbatim}
Then after the \verb+\begin{document}+ command type:
\begin{verbatim}
\maketitle
\end{verbatim}
\end{frame}

\begin{frame}[fragile]{Contents etc...}
Based on your chapters, sections, subsections, subsubsections:
\begin{verbatim}
\tableofcontents
\end{verbatim}
If you have figures and tables you can also produce
\begin{verbatim}
\listoftables
\listoffigures
\end{verbatim}
\end{frame}

\subsection{Labels}
\begin{frame}[fragile]{Numbers}
Many environments produce numbers:\\
(e.g. \verb+\section, \begin{equation}+ \verb+\begin{enumerate}, \begin{table} +)\\[0.5cm]
If it is numbered it can be {\em labelled} and {\em referred to} :
\begin{verbatim}
\section{A subsection} \label{seclabelex}
\begin{equation}
x=y^2 \label{eq1}
\end{equation}
\end{verbatim}
Then later in the text:
\begin{verbatim}
In equation (\ref{eq1}) in subsection \ref{seclabelex} 
on page \pageref{intro} we discussed....
\end{verbatim}
\end{frame}

\subsection{A subsection} \label{seclabelex}
\begin{frame}[fragile]{Output}
\begin{equation}
x=y^2 \label{eq1}
\end{equation}
In equation (\ref{eq1}) in subsection \ref{seclabelex} 
on page \pageref{seclabelex} we discussed....
\end{frame}

\subsection{BibTeX}
%% A separate .bib file for bibliography - how to use it
%% Syntax
%% How to input in document

\begin{frame}[fragile]{The bibliography}
The notes explain how to use a simple within-document bibliography.\\[0.5cm]

My advice: record anything you've ever read in a separate \emph{Bib\TeX} file.\\[0.5cm]

References will only appear if they are cited in the current document.
\end{frame}

\begin{frame}[fragile]{A Bib\TeX\ file}
... should finish with .bib. Example syntax:
{\scriptsize
\begin{verbatim}
@Article{LillyPark,
author={Jonathan Lilly and Jeffrey Park},
title={Multiwavelet Spectral and Polarization Analysis of Seismic Records},
journal={Geophysical Journal International},
year={1995},
volume={122},
pages={1001--1021}
}

@Book{Daub,
author={Ingrid Daubechies},
title={Ten Lectures on Wavelets},
publisher={SIAM Press},
year={1992},
address={Philadelphia, USA}
}
\end{verbatim}
}
\end{frame}

\begin{frame}[fragile]{Entry types}
Example entry types: \verb+article+, \verb+book+, \verb+manual+, \verb+phdthesis+, \verb+inproceedings+, any many more.\\[0.5cm]

Each has its own mandatory and optional fields.\\[0.5cm]

See e.g.
\begin{verbatim}
http://en.wikipedia.org/wiki/BibTeX
\end{verbatim}
\end{frame}

\begin{frame}[fragile]{Placing and citing in document}
Just before \verb+\end{document}+:
\begin{verbatim}
\bibliographystyle{plain}
\bibliography{name}
\end{verbatim}
To cite in the document, use e.g. \verb+\cite[p.12]{label}+.
\end{frame}

\begin{frame}{Compiling}
This depends on the editor but traditionally: 
\begin{itemize}
\item whenever the global numbering has changed (e.g. you have added a new section), \LaTeX\ needs to be compiled twice.
\item whenever you input a new reference, compile \LaTeX\ once, then BiB\TeX\ once, then \LaTeX\ twice!
\end{itemize}
\end{frame}

\subsection{Counters}
\begin{frame}[fragile]{To change numbering}
Use the following {\em counters}:
{\scriptsize
\begin{tabbing}
numbering: \hspace{2cm} \= page \\
\> chapter \\
\> section, subsection \\
\> equation \\
\> figure \\
\> table \\

For enumerate: \> enumi \\
\> enumii \\
\> enumiii \\
\> enumiv \\
\end{tabbing}
\begin{verbatim}
\setcounter{section}{5}
\addtocounter{section}{-2}
\end{verbatim}
}
\end{frame}

\begin{frame}[fragile]{Printing counter numbers}
\begin{verbatim}
\setcounter{page}{7}
\arabic{page}
\roman{page}
\Roman{page}
\alph{page}
\Alph{page}
\end{verbatim}
produces:\setcounter{page}{7}
\arabic{page}
\roman{page}
\Roman{page}
\alph{page}
\Alph{page}
\\
\setcounter{page}{56}
To change numbering, add a \verb+\the+ to the front of the counter name, e.g. to relabel the 4th subsection of the 2nd section ``II-D'':\\
{\footnotesize
\begin{verbatim}
\renewcommand{\thesection}{\Roman{section}}
\renewcommand{\thesubsection}{\thesection-\Alph{subsection}}
\end{verbatim}
}
\end{frame}

%%Other things: include and input
%%style files and
\begin{frame}[fragile]{Input and include}
To split a lot of code into multiple files use \verb+\input+, e.g.
\begin{verbatim}
\input{chap1}
\input{chap2}
\input{chap3}
\end{verbatim}
If you only want to print part of the document, use \verb+\include+, e.g. to only print chapters 2 and 3:
\begin{verbatim}
\includeonly{chap2,chap3}
\documentclass{article}
\begin{document}
\include{chap1}
\include{chap2}
\include{chap3}
\end{document}
\end{verbatim}
\end{frame}

\begin{frame}[fragile]{Exercise 3}
\begin{verbatim}
http://www2.imperial.ac.uk/~ejm/LaTeX/
          Website/exercises/exercise3.html
\end{verbatim}
\end{frame}
\end{document}