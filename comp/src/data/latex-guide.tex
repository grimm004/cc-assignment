\documentclass{article}

\usepackage{amsmath, amsfonts, amsthm, fullpage, amssymb}
\input{macros}

\pagestyle{plain}

\parskip=1.5mm
\parindent=0mm

\title{A Beginner's Guide to \LaTeX}
\author{David Xiao \\
\texttt{dxiao@cs.princeton.edu}}

% These are some useful macros

\begin{document}

% This prints out the document's title, author, and date
\maketitle

\section{Introduction}
\LaTeX{} is \emph{the} standard mathematical typesetting program.  This
document is for people who have never used \LaTeX{} before and just want
a quick crash course to get started.  I encourage all students in
mathematics and theoretical computer science to learn \LaTeX{} so you
can use it to typeset your problem sets; your TA's will love you if
you do.

For a more comprehensive introduction, check out \\
\texttt{http://ctan.tug.org/tex-archive/info/lshort/english/lshort.pdf}.

\section{How to find and use \LaTeX?}
There are Windows packages for \LaTeX, though typically I prefer to
use it on a Unix system (e.g. OS X, Linux, Solaris).  You will find
that most university clusters have \LaTeX{} installed.  If you prefer,
you can install it on your home system; it is easy to find via Google
or your search engine of choice.  I will assume that you are using a
Unix system for the rest of the guide.

I recommend using an editor like \texttt{emacs} or \texttt{vi} to edit
\LaTeX.  If you are uncomfortable with these, then any plaintext
editor will do.  \texttt{emacs} is particularly nice because it has a
built-in \LaTeX-mode that does text highlighting and indentation.

\section{Basic rules}
Basic \LaTeX{} is just text with typesetting commands.  Typesetting
commands are usually preceded by ``\verb1\1'', and any arguments
are usually placed inside curly braces ``\verb1{}1''.

\LaTeX{} wraps text in adjacent lines as if they were part of the same
paragraph.  To start a new paragraph, insert an extra ``return'':

\begin{tabular}{ll}
Source: & Output: \\
\begin{minipage}{.5 \linewidth}
\begin{verbatim}
This is one paragraph.

This is another.
\end{verbatim}
\end{minipage}
&
\framebox{\begin{minipage}{.5 \linewidth}
\parskip=1.5mm
This is one paragraph.

This is another.
\end{minipage}}
\end{tabular}

\bigskip

To get a newline without starting a new paragraph, use \verb1\\1.

To get a comment, use the percent sign \verb1%1 at the beginning of a
line.  The rest of that particular line will be commented out.

\section{Starting a new document}

The most basic (empty) document has only three parts:
\begin{verbatim}
\documentclass{article}

\begin{document}
\end{document}

\end{verbatim}

To start a new document, you can just take the \LaTeX{} file for this
document and delete the stuff between the begin and end document
commands, leaving the \verb1\maketitle1 command (this prints out the
title and your name).  You will see before the \verb1\begin{document}1
that there are commands for the title and author of the document.
Change the names between the curly braces to the name of problem set
and your name.  (The other stuff that precedes \verb1\begin{document}1
is useful stuff that you don't need to worry about for now, just leave
it as is.)  Then, save the file under a different name, for example
\texttt{problem-set-1.tex}.

\section{Compiling}
Suppose our file is named \texttt{mylatexfile.tex}.  To compile it,
simply invoke \texttt{latex mylatexfile.tex} in your Unix shell.  This
will compile the file, assuming there are no errors.\footnote{You may
have to invoke \texttt{latex} twice if you are using labels and
references.  See Section \ref{sec:labels}.} If there are errors, you
can quit the compiler by hitting ``\texttt{x}'' and then enter.
Unfortunately \LaTeX{} compiler errors are very unhelpful in determining
the nature of the problem, but they usually correctly point you to the
line where the error occurred.

Once it successfully compiles, you will get a file named
\texttt{mylatexfile.dvi}.  Typically we convert this either into a
Postscript or PDF file, which may be done by the programs
\texttt{dvips mylatexfile.dvi} and \texttt{dvipdf mylatexfile.dvi}.
Sometimes \texttt{dvips} requires an extra option: \texttt{-o}
\textit{output-file-name} specifying the output file.

\section{Organization}
One important thing to do is to organize your document well.

\subsection{Sectioning}
There are two sectioning commands that will be useful for you:
\verb1\section{Name of section}1 and\\
\verb1\subsection{Name of subsection}1.  Use these to separate
different problems or subproblems in the assignment.

\subsection{Tables}

You can put stuff into tables by using the \texttt{tabular}
environment.  For example:

\begin{tabular}{ll}
Source: & Output: \\
\begin{minipage}{.5 \linewidth}
\begin{verbatim}
\begin{tabular}{r|cl}
1st column & 2nd column & 3rd column\\
\hline
a & b & c
\end{tabular}
\end{verbatim}
\end{minipage}
&
\framebox{\begin{minipage}{.5 \linewidth}
\parskip=1.5mm
\begin{tabular}{r|cl}
1st column & 2nd column & 3rd column\\
\hline
a & b & c
\end{tabular}
\end{minipage}}
\end{tabular}

Note that the command is called \texttt{tabular} and \emph{not}
\texttt{table}.  Important points:
\begin{itemize}
\item The ``\verb1{r|cl}1'' after the tabular \verb1\begin{tabular}1
  indicate the alignment of the three columns: right, center, and
  left.  This is mandatory as it specifies the layout of the table.
  For more columns, type more alignment commands, e.g. for a table
  with 5 columns all aligned to the right, you would use
  \verb1rrrrr1. 
\item The vertical bar \verb1|1 between the \texttt{r} and \texttt{c}
indicates that a vertical line should be drawn between those columns.'
\item The \verb1&1 separates the columns in the body of the table.
\item A \verb1\\1 signifies the end of each line of the table.
\item The command \verb1\hline1 means that a horizontal line should be
  inserted.
\end{itemize}

\subsection{Lists}

You can put stuff into ordered and unordered lists by using the
\texttt{enumerate} and \texttt{itemize} commands, respectively.  For
example:

\begin{tabular}{ll}
Source: & Output: \\
\begin{minipage}{.5 \linewidth}
\begin{verbatim}
Unordered list:

\begin{itemize}
\item This is one item.
\item This is another.
\end{itemize}

Ordered list:

\begin{enumerate}
\item This is the first item.
\item This is the second.
\end{enumerate}
\end{verbatim}
\end{minipage}
&
\framebox{\begin{minipage}{.5 \linewidth}
\parskip=1.5mm
Unordered list:

\begin{itemize}
\item This is one item.
\item This is another.
\end{itemize}

Ordered list:

\begin{enumerate}
\item This is the first item.
\item This is the second.
\end{enumerate}
\end{minipage}}
\end{tabular}

\subsection{Labels and references}
\label{sec:labels}
It is useful to refer to the section number at times.  This may be
done by using the \verb1\label{labelname}1 command.  Place this right
after you start a section.  Then, you may refer to the section number
by using \verb1\ref{labelname}1.  This will also be useful to refer to
math equations.

Note that \LaTeX creates and uses a bunch of auxiliary files.  Thus
you will have to invoke it twice to compile a file that has labels and
references, or if thoses labels and references have changed since the
last compilation.

\section{Math}
The reason we use \LaTeX{} is because it is so powerful in typesetting
mathematical expressions.  It wins hands down versus word processors
like Word.

\subsection{Math mode}
Math expressions are separate from text in \LaTeX.  To enter a math
environment in the middle of text, use the dollar sign \verb1$1, for
example \verb1$F = ma$1 produces $F = ma$.  Everything between the two
\verb1$1 signs will be considered math formula.

To type a math expression that is on its own line and centered, use
\verb1$$1:

\begin{tabular}{ll}
Source: & Output: \\
\begin{minipage}{.5 \linewidth}
\begin{verbatim}
The following is an important equation:
$$E = mc^2$$
\end{verbatim}
\end{minipage}
&
\framebox{\begin{minipage}{.5 \linewidth}
The following is an important equation:
$$E = mc^2$$
\end{minipage}}
\end{tabular}

\bigskip

To give an equation a number and have it referable, use the
\verb1equation1 environment and use a \verb1\label1 command:

\begin{tabular}{ll}
Source: & Output: \\
\begin{minipage}{.5 \linewidth}
\begin{verbatim}
The following is an important equation:
\begin{equation}
\label{emc}
E = mc^2
\end{equation}
Please memorize Equation \ref{emc}.
\end{verbatim}
\end{minipage}
&
\framebox{\begin{minipage}{.5 \linewidth}
The following is an important equation:
\begin{equation}
\label{emc}
E = mc^2
\end{equation}
Please memorize Equation \ref{emc}.
\end{minipage}}
\end{tabular}

\bigskip

To typeset several equations together and have them properly aligned,
use the \verb1align1 environment:

\begin{tabular}{ll}
Source: & Output: \\
\begin{minipage}{.5 \linewidth}
\begin{verbatim}
Some important equations:
\begin{align}
\label{einstein}
E & = mc^2 \\
\label{newton}
F & = ma \\
\label{euler}
e^{i \pi} & = -1 
\end{align}
\end{verbatim}
\end{minipage}
&
\framebox{\begin{minipage}{.5 \linewidth}
Some important equations:
\begin{align}
\label{einstein}
E & = mc^2 \\
\label{newton}
F & = ma \\
\label{euler}
e^{i \pi} & = -1 
\end{align}
\end{minipage}}
\end{tabular}
The equations are aligned along the \verb1&1 and each line is
terminated by \verb1\\1.  To suppress the equation numbering (i.e. if
the equations won't be referred to) use \verb1align*1 instead of
\verb1align1.

\subsection{Writing math expressions}
I will only go over a few common mistakes and hard-to-find expressions
regarding how to write math expressions.  It is quite intuitive
otherwise, you can figure most things out quickly with trial and
error.  All expressions in math mode may be nested within each other
arbitrarily.

\begin{itemize}
\item Superscript and subscript are done using \verb1^1 and \verb1_1
characters.  Note that if you want multiple characters in the
super/subscript then you need to surround them with curly braces:
\verb2$e^i\pi = -1$2 gives $e^i\pi = -1$ whereas
\verb2$e^{i\pi} = -1$2 gives $e^{i\pi} = -1$.

\item Fractions are done using \verb3$\frac{1}{2}$3 which gives
  $\frac{1}{2}$.
\item To do a binomial coefficient, use \verb1$\binom{n}{k}$1 which
  gives $\binom{n}{k}$.
\item Modular arithmetic can be written using the \verb1\pmod{n}1 and
  \verb1\bmod{n}1 commands.  The first puts parentheses and a lot of
  space around the \texttt{mod} and the second does not.
\item $\forall$ and $\exists$ are written as \verb1\forall1 and
  \verb1\exists1.
\item $\neq$, $\geq$, and $\leq$ are \verb1\neq1, \verb1\geq1, and
  \verb1\leq1.
\item $\cdot$ (e.g. for multiplication) is \verb1\cdot1.
\item $\circ$ is \verb1\circ1.
\item $\cup$ and $\cap$ are \verb1\cup1 and \verb1\cap1.
\item $\oplus$ is written with \verb1\oplus1.
\item Large $\cup$, $\cap$, $\oplus$ signs that behave like summations
  (see below for summations) are written as \verb1\bigcup1,
  \verb1\bigcap1, \verb1\bigoplus1.
\item $\drawnr$ is produced with \verb1\drawnr1.
\item $\Z$, $\R$, etc. are produced using \verb1\Z1, \verb1\R1, etc.
\item $\Exp$ is produced with \verb1\Exp1.
\item $\P$, $\NP$, etc. are produced using \verb1\P1, \verb1\NP1, etc.
\item $\calP$ is \verb1\calP1.
\item $\ell$ (as opposed to $l$) is produced with \verb1\ell1.
\item $\{\}$ are done with \verb1\{1 and \verb1\}1.
\item $\approx$ is produced with \verb1\approx1.
\item $\hat{x}$ and $\bar{x}$ are done with \verb1\hat{x}1 and
  \verb1\bar{x}1.  A longer bar may be written using
  \verb1\overline{\SAT}1, which produces $\overline{\SAT}$.
\item $\eps$ (as opposed to $\epsilon$) may be written with
  \verb1\eps1.
\item $\in$ is written as \verb1\in1.
\item $\notin$ is written as \verb1\notin1.
\item Negations may be done with \verb1\not1, for example
\item $\zo$ is abbreviated as \verb1\zo1.
  \verb1\not\geq1 gives $\not\geq$.
\item The probability sign is defined as \verb1\Pr1, i.e. $\Pr$.
\item Encryption and decryption are \verb1\Enc1 and \verb1\Dec1, which
  give $\Enc$ and $\Dec$.
\item $\poly$ is written as \verb1\poly1.
\item To draw parentheses that grow to match the contents, use
  \verb1\left1 to precede the left parenthesis and \verb1\right1 to
  precede the right:\\
Source:
\begin{verbatim}
$$\Pr\left[\sum_{i=1}^k X_i > c \right] \leq 2^{-\Omega(c^2 k)}$$
\end{verbatim}
Output:
$$\Pr\left[\sum_{i=1}^k X_i > c \right] \leq 2^{-\Omega(c^2 k)}$$

\item Arrays are like tables, except they must be used in place of
  tables when in math mode: instead of using \verb1\begin{tabular}1
  and \verb1\end{tabular}1 use \verb1\begin{array}1 and
  \verb1\end{array}1.  Again, you must give a column specification for
  how the columns are to be laid out.

\item Spacing is very different in math mode so text in the middle of
a formula is set strangely.  If you want to have text in the middle of
the formula, use the \verb1\text{some text}1 command.  For example,\\
\verb1$\P \neq \NP \text{ implies that } \SAT \notin \P$1 produces $\P
\neq \NP \text{ implies that } \SAT \notin \P$.

\item Summations and products are done using \verb1\sum1 and
  \verb1\prod1 respectively.  Parameters can be given for the
  summation/product as well:

\begin{tabular}{ll}
Source: & Output: \\
\begin{minipage}{.5 \linewidth}
\begin{verbatim}
$$\sum_{i=1}^\infty \frac{1}{2^i} = 1$$
\end{verbatim}
\end{minipage}
&
\framebox{\begin{minipage}{.5 \linewidth}
$$\sum_{i=1}^\infty \frac{1}{2^i} = 1$$
\end{minipage}}
\end{tabular}

\item Piecewise functions may be defined using the \verb1piecewise1
  environment:

\begin{tabular}{ll}
Source: & Output: \\
\begin{minipage}{.5 \linewidth}
\begin{verbatim}
$$f(x) = \begin{piecewise}
1 & x = 0 \\
0 & \text{else}
\end{piecewise}$$
\end{verbatim}
\end{minipage}
&
\framebox{\begin{minipage}{.5 \linewidth}
$$f(x) = \begin{piecewise}
1 & x = 0 \\
0 & \text{else}
\end{piecewise}$$
\end{minipage}}
\end{tabular}

\item You may define new commands using
  \verb1\newcommand{\commandname}{definition}1.  This is essentially
  a macro, so that whenever \verb1\commandname1 appears, the text of
  \verb1definition1 is inserted in its place.  If \verb1\commandname1
  is already taken, either use a different name or use
  \verb1\renewcommand{...}{...}1 to overwrite the old definition.
\item There are many many other symbols available.  You can search for
  ``latex symbols'' online and come up with the references.
\end{itemize}

\end{document}