\documentclass[12pt,letterpaper]{article}
\usepackage{latexsym}
\usepackage[dvips]{graphics}
\usepackage{epsfig}
\input amssym.def
\input amssym.tex

\newcommand\st{\mbox{s.t.\ }}
\newcommand\be{\begin{equation}}
\newcommand\ee{\end{equation}}
\newcommand\bea{\begin{eqnarray}}
\newcommand\eea{\end{eqnarray}}
\newcommand\bi{\begin{itemize}}
\newcommand\ei{\end{itemize}}
\newcommand\ben{\begin{enumerate}}
\newcommand\een{\end{enumerate}}
\newcommand\bc{\begin{center}}
\newcommand\ec{\end{center}}
\newcommand\ba{\begin{array}}
\newcommand\ea{\end{array}}
\newcommand\mod{\mbox{mod\ }}
\newcommand\lag[2]{\ensuremath{\left(\frac{#1}{#2}\right)}}
\newcommand\ie{{\it i.e.\ }}


\def\notdiv{\ \mathbin{\mkern-8mu|\!\!\!\smallsetminus}}
\newcommand{\Qoft}{\Bbb{Q}(t)}  %use in linux
\newcommand{\done}{\Box} %use in linux
\newcommand{\R}{\ensuremath{\Bbb{R}}}
\newcommand{\C}{\ensuremath{\Bbb{C}}}
\newcommand{\Z}{\ensuremath{\Bbb{Z}}}
\newcommand{\Q}{\Bbb{Q}}
\newcommand{\F}{\Bbb{F}}



%\begin{document}
%\title{Math Honors 1 : Review for lecture 1, 9/4/02}
%\author{Peter Sarnak, Steve Miller, and Alex Barnett}
%\maketitle


\title{Introduction to Some of Group Theory}

\author{Steven Miller\thanks{E-mail:
\texttt{millerj@cims.nyu.edu or sjmiller@math.princeton.edu}}}

\date{}

\begin{document}

\maketitle %\centerline{Courant Institute of Mathematics}
 %\centerline{New York University}
 %\centerline{New York, NY}

\begin{abstract}We will review some group theory.
\end{abstract}


\section{Lagrange's Theorem}

\subsection{Basic group theory}

Group $G$ is a set of elements $g_i$ satisfying the four
conditions below, relative to some binary operation. We often use
multiplicative notation ($g_1 g_2$) or additive notation ($g_1 +
g_2$) to represent the binary operation. For definiteness, we use
multiplicative notation below; however, one could replace $xy$
with $b(x,y)$ below.

If the elements of $G$ satisfy the following four properties, then
$G$ is a group.

\ben
\item $\exists e \in G \;\st \forall g \in G : eg = ge = g$. (Identity.)
We often write $e=1$ for multiplicative groups, and $e = 0$ for
additive groups.

\item $\forall x,y,z \in G$ : $(xy)z = x(yz)$. (Associativity.)

\item $\forall x \in G, \exists y \in G \;\st xy = yx = e$. (Inverse.)
We write $y = x^{-1}$ for multiplication, $y = -x$ for addition.

\item $\forall x,y \in G : xy \in G$. (Closure.)
\een

If commutation holds ($\forall x, y \in G$, $xy = yx$), we say the
group is Abelian. Non-abelian groups exist and are important. For
example, consider the group of $N \times N$ matrices with real
entries and non-zero determinant. Prove this is a group under
matrix multiplication, and show this group is not commutative.

$H$ is a {\em subgroup} of $G$ if it is a group and its elements
form a subset of those of $G$. The identity of $H$ is the same as
the identity of $G$. Once you've shown the elements of $H$ are
closed (ie, under the binary operation, $b(x,y) \in H$ if $x, y
\in H$), then associativity in $H$ follows from closure in $H$ and
associativity in $G$.

For the application to Fermat's Little Theorem you will need to
know that the set $\{1,x,x^2,\cdots\,x^{n-1}\}$ where $n$ is the
lowest positive integer \st $x^n = 1$, called the {\em cyclic
group}, is indeed a subgroup of any group $G$ containing $x$, as
well as $n$ divides the order of $G$.

For a nice introduction to group theory see: M. Tinkham, {\em
Group Theory and Quantum Mechanics}, (McGraw-Hill, 1964) or S.
Lang, {\em Undergraduate Algebra}.


\subsection{Lagrange's Theorem}

The theorem states that if $H$ is a subgroup of $G$ then $|H|$ divides
$|G|$.

First show that the set $hH$, \ie all the elements of $H$ premultiplied
by one element, is just $H$ rearranged (Cayley's theorem).
By closure $hH$ falls within $H$.
We only need to show that
$h h_i$ can never equal $h h_j$ for two different elements $i\ne j$.
If it were true, since a unique $h^{-1}$ exists we could premultiply
the equation $h h_i = h h_j$ by
$h^{-1}$ to give $h_i = h_j$, which is false. Therefore $h h_i \ne h h_j$,
and we have guaranteed a 1-to-1 mapping from $H$ to $hH$, so $hH = H$.

Next we show that the sets $g_i H$ and $g_j H$ must either be
completely disjoint, or identical. Assume there is some element in
both. Then $g_i h_1 = g_j h_2$. Multiplying on the right by
$h_i^{-1} \in H$ (since $H$ is a subgroup) gives $g_i = g_j h_2
h_1^{-1}$. As $H$ is a subgroup, $\exists h_3 \in H$ such that $h
= h_2 h_1^{-1}$. Thus $g_i = g_j h_3$. Therefore, as $h_3 H = H$,
$g_i H = g_j h_3 H = g_j H$, and we see if the two sets have one
element in common, they are identical. We call a set $gH$ a
\emph{coset} (actually, a left coset) of $H$.

Clearly

\be
G = \bigcup_{g \in G} g H \ee

Why do we have an equality? As $g \in G$ and $H \subset G$, every
set on the right is contained in $G$. Further, as $e \in H$, given
$g \in G$, $g \in gH$. Thus, $G$ is a subset of the right side,
proving equality.

There are only finitely many elements in $G$. As we go through all
$g$ in $G$, we see if the set $gH$ equals one of the sets already
in our list (recall we've shown two cosets are either identical or
disjoint). If the set equals something already on our list, we do
not include it; if it is new, we do. Continuing this process, we
obtain

\be
G = \bigcup_{i = 1}^k g_i H
\ee

for some finite $k$. If $H = \{e\}$, $k$ is the number of elements
of $G$; in general, however, $k$ will be smaller.

Each set $g_i H$ has $|H|$ elements. Thus, $|G| = k|H|$, proving
$|H|$ divides $|G|$.


\section{Quotient groups}

Say we have a finite Abelian group $G$ (this means for all $x, y
\in G$, $xy = yx$) of order $m$ which has a subgroup $H$ of order
$r$. We will use multiplication as our group operation. Recall the
{\em coset} of an element $g\in G$ is defined as the set of
elements $gH = g\{h_1,h_2,\cdots,h_r\}$. Since $G$ is Abelian
(commutative) then $gH = Hg$ and we will make no distinction
between left and right cosets here.

The {\em quotient group} (or {\em factor group}), symbolized by
$G/H$, is the group formed from the cosets of all elements $g\in
G$. We treat each coset $g_i H$ as an element, and define the
multiplication operation as usual as $g_i H g_j H$. Why do we need
$G$ to be Abelian? The reason is we can then analyze $g_i H g_j
H$, seeing that it equals $g_i g_j H H$. We will analyze this
further when we prove that the set of cosets is a group.

There are several important facts to note. First, if $G$ is not
Abelian, then the set of cosets might not be a group. Second,
recall we proved the coset decomposition rule: given a finite
group $G$ (with $n$ elements) and a subgroup $H$ (with $r$
elements) then there exist elements $g_1$ through $g_k$ such that

\be
G = \bigcup_{i=1}^k g_i H. \ee

The choices for the $g_i$'s is clearly not unique. If $g_1$
through $g_k$ work, so do $g_1 h_1$ through $g_k h_k$, where $h_i$
is any element of $H$. Recall this was proved by showing any two
cosets are either distinct or identical.

We will show below that, for $G$ Abelian, the set of cosets is a
group. Note, however, that while it might at first appear that
there are many different ways to write the coset group, they
really are the same. For example, the cosets $gH$ and $g h_1 h_2^4
h_3 H$ are equal. This is similar to looking at integers mod $n$;
mod $12$, the integers $5$, $-7$ and $19$ are all equal, even
though they look different.

We now prove that the set of cosets is a group (for $G$ Abelian).


{\bf Closure.} By commutivity $g_i H g_j H = g_i g_j H H$. What is
``$H H$''? Just the set of all $r^2$ possible combinations of
elements of $H$. By closure, and the existence of the identity,
this just gives $H$ again (recall no element in a group can appear
more than once---duplicates are removed). Therefore $g_i H g_j H =
g_i g_j H$. Now, as $G$ is a group and is closed, $g_i g_j \in G$.
Thus, there is a $\alpha$ such that $g_i g_j \in g_\alpha H$ (as
$G = \bigcup_{\beta=1}^k g_\beta H$. Therefore, there is an $h \in
H$ such that $g_i g_j = g_\alpha h$, which implies $g_i g_j H =
g_\alpha h H = g_\alpha H$. Thus, the set of cosets is closed
under coset multiplication. Note, however, that while the coset
$g_i g_j H$ is in our set of cosets, it may be written
differently.

{\bf Identity.} If $e$ is identity of $G$, then $eH g_i H = g_i H$
and $g_i H eH = g_i H$, so $eH$ is the identity of this quotient
group.

{\bf Associativity.} Since as you may have noticed, the quotient
group elements behave just like those of $G$, associativity
follows from that of $G$.

{\bf Inverse.} It is easy to guess $g^{-1} H$ is the inverse of $g
H$. Check it: $g^{-1} H g H = g^{-1} g H = eH = $ identity, also
true the other way round of course by commutativity.
Unfortunately, $g^{-1} H$ might not be listed as one of our
cosets! Thus, we must be a little more careful. Fortunately, as
$g^{-1} \in G = \bigcup_{\beta=1}^k g_\beta H$, there is an
$\alpha$ such that $g^{-1} \in g_\alpha H$. Then, there is an $h
\in H$ with $g^{-1} = g_\alpha h$. Thus, $g^{-1} = g_\alpha h H =
g_\alpha H$, and direct calculation will show that the coset
$g_\alpha H$ is the inverse (under coset multiplication) of $g H$.




\end{document}
