\hypertarget{w0001----archive-as-package-file-detected}{%
\subsection{W0001 -\/- Archive as package file
detected}\label{w0001----archive-as-package-file-detected}}

Usually a CTAN package should not contain archives. An exception are
situations where, for example, the source code of a package is kept in a
separate zip archive.

\hypertarget{w0002----duplicate-files-detected}{%
\subsection{W0002 -\/- Duplicate files
detected}\label{w0002----duplicate-files-detected}}

Duplicate files were detected which are listed right after this message.

The message is a warning message as something like this could not be
seen as an error in general.

\hypertarget{w0003----same-named-files-detected-in-the-package-tree}{%
\subsection{W0003 -\/- Same named files detected in the package
tree}\label{w0003----same-named-files-detected-in-the-package-tree}}

We like to have unique file names over the whole package directory tree.
When we discover same named files we report it as a warning. Common
names like \texttt{README}, \texttt{README.txt}, \texttt{README.md},
\texttt{Makefile}, \texttt{Makefile.in}, \texttt{Makefile.am} and
\texttt{makefile} are ignored when checking.

For more details refer to:
\href{http://mirror.utexas.edu/ctan/help/ctan/CTAN-upload-addendum.html\#uniquefilenames}{http://mirror.utexas.edu/ctan/help/ctan/CTAN-upload-addendum.html\#uniquefilenames}

\hypertarget{w0004-----encoding-with-bom-detected}{%
\subsection{\texorpdfstring{W0004 -\/- encoding with BOM
detected}{W0004 -\/-  encoding with BOM detected}}\label{w0004-----encoding-with-bom-detected}}

A UTF encoded package file contains a BOM (byte order mark). Currently,
we issues a warning.

Nevertheless, the CTAN team discourages uses of BOM. Please be aware,
that in some future time this could be reagarded as an error.

\hypertarget{w0005----very-large-file--with-size-size-detected-in-package}{%
\subsection{\texorpdfstring{W0005 -\/- Very large file with size
\texttt{\textless{}size\textgreater{}} detected in
package}{W0005 -\/- Very large file  with size \textless size\textgreater{} detected in package}}\label{w0005----very-large-file--with-size-size-detected-in-package}}

(Experimental) We issue the message if there is a file is larger than
40MiB in the package directory tree.

\hypertarget{w0006----very-large-file-with-size-size-detected-in-tds-zip-archive}{%
\subsection{\texorpdfstring{W0006 -\/- Very large file with size
\texttt{\textless{}size\textgreater{}} detected in TDS zip
archive}{W0006 -\/- Very large file with size \textless size\textgreater{} detected in TDS zip archive}}\label{w0006----very-large-file-with-size-size-detected-in-tds-zip-archive}}

(Experimental) We issue the message if there is a file larger than 40MiB
in the TDS zip archive.

\hypertarget{w0007----empty-directory-detected-in-the-tds-zip-archive}{%
\subsection{W0007 -\/- Empty directory detected in the TDS zip
archive}\label{w0007----empty-directory-detected-in-the-tds-zip-archive}}

Empty directories in a TDS zip archive are discouraged. As they usually
don't create errors in the distribution we issue a warning only.

\hypertarget{w0008----windows-file-has-unix-line-endings}{%
\subsection{W0008 -\/- Windows file has Unix line
endings}\label{w0008----windows-file-has-unix-line-endings}}

A Windows file with Unix line endings was detected.

We regard a file as a Windows file if its name ends with:

\begin{itemize}
\tightlist
\item
  \texttt{.bat}
\item
  \texttt{.cmd}
\item
  \texttt{.nsh}, or
\item
  \texttt{.reg}
\end{itemize}
