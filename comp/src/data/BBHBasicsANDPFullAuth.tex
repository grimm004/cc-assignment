\documentclass{andp2012}% no class options needed by now
\usepackage[english]{babel}

%%%%%%%%%%%%%%%%%%%%%%%%%%%%%%
%%% The following commands will be completed by the publisher.  Please leave 
%%% these to the editors.
% \setcopyrightyear{2012}%
% \DOIprefix{10.1002}%
% \DOIsuffix{andp.201100xxx}%
% \Volume{524}%
% \Issue{}%
% \Month{}%
% \Year{2012}%
%%\Day{1}
% \pagespan[A]{}{}% page numbers are irrelevant for now
% \Receiveddate{}
% \Reviseddate{}
% \Accepteddate{}
% \Dateposted{}
%%%%%%%%%%%%%%%%%%%%%%%%%%%%%%

\usepackage{etoolbox}
\usepackage{url,hyperref}


%%%%%%%%%%%

\def\w{\omega}
\def\D{{\rm d}} 
\def\e{{\rm e}} 
\def\i{{\rm i}} 

\def\({\left(}
\def\){\right)}
\def\be{\begin{equation}}
\def\ee{\end{equation}}
\def\bea{\begin{eqnarray}}
\def\eea{\end{eqnarray}}
\def\ergo{\Longrightarrow}
%\newcommand{\inlinecite}[1]{Ref.~\onlinecite{#1}}
%\newcommand{\inlinecites}[1]{Refs.~\onlinecite{#1}}
\newcommand{\inlinecite}[1]{\cite{#1}}
\newcommand{\inlinecites}[1]{\cite{#1}}
\newcommand{\refigure}[1]{Fig.~\ref{#1}}
\newcommand{\refigures}[1]{Figs.~\ref{#1}}
\newcommand{\refapp}[1]{App.\ \ref{#1}}
\newcommand{\refsec}[1]{Sec.\ \ref{#1}}
\newcommand{\refeqn}[1]{Eq.\ \ref{#1}}
%\newcommand{\caveatformat}[1]{{\bf #1}. }
\newcommand{\caveatformat}[1]{\subsection{#1}}

\def\Mc{\mathcal{M}}
\def\Lplanck{L_{\rm Planck}}

\def\fgw{f_{\rm GW}}
\def\forb{f}
\def\wgw{\w_{\rm GW}}
\def\worb{\w}

\def\submax{\big|_{\rm max}}
\def\fgwmax{f_{\rm GW} \submax}
\def\subfin{\big|_{\rm fin}}
\def\fgwfin{f_{\rm GW} \subfin}
\def\subRD{|_{\rm ringdown}}
\def\fRD{{\fgw}\subRD}

\def\wkep{\w_{\rm Kep}}
\def\wkepmax{\wkep\submax}
\def\orb{_{\rm orb}}

\def\hmax{h\submax}

%%% use acronyms
%\def\BH{BH }		\def\BHs{BHs }		\def\BHns{BH}		\def\BHsns{BHs}		\def\GW{GW }		\def\GWs{GWs }				\def\GWns{GW}		\def\GWsns{GWs}		\def\LR{LR }		\def\ISCO{ISCO }		\def\NS{NS }		\def\NSs{NSs }			\def\GR{GR }

%%% expand out acronyms
\def\BH{black hole }		\def\BHs{black holes }			\def\BHns{black hole}		\def\BHsns{black holes}		\def\GW{gravitational wave }		\def\GWs{gravitational waves }				\def\GWns{gravitational wave}		\def\GWsns{gravitational waves}		\def\LR{light ring }		\def\ISCO{innermost stable circular orbit }		\def\NS{neutron star }		\def\NSs{neutron stars }		\def\GR{general relativity }

%%% some macros used by the detection paper for standardization

\newcommand{\Msun}{\ensuremath{\mathrm{M}_\odot\,}}

\newcommand{\MCOMPONENTAPPROX}{\ensuremath{35\, \Msun}}
\newcommand{\CHIRPSTRAINPEAK}{\macro{\ensuremath{1.0 \times 10^{-21}}}} 
\newcommand{\CHIRPFSTRAINPEAK}{\ensuremath{150}} % Hz
\newcommand{\MFINALAPPROX}{\ensuremath{70\, \Msun}}

\newcommand{\CHIRPMASSAPPROX}{\ensuremath{30\, \Msun}}
\newcommand{\CHIRPFMIN}{\macro{\ensuremath{30}}} % Hz
\newcommand{\CHIRPFATMAX}{\macro{\ensuremath{170}}} % Hz
\newcommand{\CHIRPFMAX}{250} % Hz
\newcommand{\CHIRPDURATION}{\macro{\ensuremath{0.2}}} % s
\newcommand{\ORBITALSEP}{\ensuremath{347}} % km
\newcommand{\ORBITALSEPAPPROX}{\ensuremath{350}} % km





%% Enable/disable the full author list
\newtoggle{fullauthorlist}
\toggletrue{fullauthorlist}
%\togglefalse{fullauthorlist}

%% Enable/disable placing author list at the end
\newtoggle{endauthorlist}
\toggletrue{endauthorlist}
%\togglefalse{endauthorlist}



\iftoggle{endauthorlist}{
%
% Put the author list at the end of the document.
% Save author, affiliation, and maketitle commands.
%
\let\mymaketitle\maketitle
\let\myauthor\author
\let\myaffiliation\affiliation
\author{The LIGO Scientific Collaboration
%\footnote{Corresponding author\quad E-mail:~\textsf{info??@ligo.org}}
}
\author{The Virgo Collaboration
\footnote{Full author list appears at the end.}
}
}{
%
% Keep the author list on the initial title page.
%
\iftoggle{fullauthorlist}{
\input{BBHBasicsAuthors.txt}
}{
\author{The LIGO Scientific Collaboration}
\affiliation{LSC}
\author{The Virgo Collaboration}
\affiliation{Virgo}
}
}



%%% You can give a category and subcategory, but you don't have to. It is 
%%% recommended to leave these lines as-is.
% \category{Review Article/Original Paper}
% \subcategory{}
%%% Please give keywords, if there are any, in a comma separated list starting
%%% with a capital and ending with a period.
\keywords{GW150914, gravitational waves, black holes.}
\title{The basic physics of the binary black hole merger GW150914}
%%% this one is not mandatory:
% \subtitle{subtitle}
%%% In two-column articles the sequence of author names and addresses in the
%%% article title page is identical to the sequence of the respective \author
%%% and \address macros here.  They don't need to be intertwined as in 
%%% three-column mode, though.  The correspondence between author names and 
%%% (probably multiple) affiliations is shown using numerical tags in \inst 
%%% macros and the optional arguments of the \address declarations, 
%%% respectively:

%\author[B. Allen]{Bruce Allen\inst{1,2,3}}
%\author[O. Birnholtz]{Ofek Birnholtz\inst{1} \footnote{Corresponding author\quad E-mail:~\textsf{ofek.birnholtz@aei.mpg.de}}}
%\author[A. B. Nielsen]{Alex B. Nielsen\inst{1}}
%\author[S. Ghosh]{Shaon Ghosh\inst{4,5}}
%\author[A. G. Wiseman]{Alan G. Wiseman\inst{3}}
%\address[1]{Max-Planck-Institut f\"ur Gravitationsphysik, D-30167 Hannover, Germany}
%\address[2]{Leibniz Universit\"at Hannover, D-30167 Hannover, Germany}
%\address[3]{University of Wisconsin-Milwaukee, Milwaukee, WI 53201, USA}
%\address[4]{Department of Astrophysics/IMAPP, Radboud University Nijmegen, P.O. Box 9010, 6500 GL Nijmegen, The Netherlands}
%\address[5]{Nikhef, Science Park, 1098 XG Amsterdam, The Netherlands}

\shortauthors{LIGO Scientific \& VIRGO Collaborations}




\begin{abstract}
The first direct gravitational-wave detection was made by the
Advanced Laser Interferometer Gravitational Wave Observatory
on September 14, 2015.
The GW150914 signal was strong enough
to be apparent,
without using any waveform model,
in the filtered detector strain data.
Here those features of the signal visible in these data are used,
along with only such concepts from Newtonian physics
and general relativity
as are accessible to anyone with a general physics background.
The simple analysis presented here is consistent with
the fully general-relativistic analyses published elsewhere,
in showing that the signal was produced by
the inspiral and subsequent merger of two black holes.
The black holes were each of approximately
$\MCOMPONENTAPPROX{}$,
still orbited each other as close as
$\sim\ORBITALSEPAPPROX$ km apart
and subsequently merged to form a single black hole.
%no other known type of astrophysical objects
%could have been responsible for the GW150914 signal.
Similar reasoning, directly from the data,
is used to roughly estimate how far these black holes were from the Earth,
and the energy that they radiated in gravitational waves.
\end{abstract}
\shortabstract

%%% Here, the document begins.
\begin{document}
\maketitle

\section{Introduction}
\label{Sec:intro}

Advanced LIGO made the first observation of a gravitational wave (GW) signal, GW150914 \cite{DetectionPaper},
on September 14th, 2015,
a successful confirmation of a prediction by
Einstein's theory of general relativity (GR).
The signal was clearly seen by the two LIGO detectors
located in Hanford, WA and Livingston, LA.
Extracting the full information about the source of the signal
requires detailed analytical and computational methods
(see \inlinecites{PEPaper, BetterPE, TestingGRPaper, AstroPaper, O1BBH}
and references therein for details).
However, much can be learned about the source
by direct inspection of the detector data
and some basic physics\cite{Schutz:1984nf},
accessible to a general physics audience,
as well as students and teachers.
This simple analysis indicates that the source is
two black holes (BHs) orbiting around one another
and then merging to form another \BH.

\begin{figure}[h!]
\centering
\includegraphics[width=\columnwidth]{figs/StrainFreqDetPaper1.jpg}
\caption{
 The instrumental strain data in the
 Livingston detector (blue) and Hanford detector (red),
 as shown in Figure 1 of \inlinecite{DetectionPaper}.
 Both have been bandpass- and notch-filtered.
 The Hanford strain has been shifted back in time by 6.9 ms and inverted.
 Times shown are relative to 09:50:45 Coordinated Universal Time (UTC) on September 14, 2015.
}
 \label{f:rawdata}
\end{figure}

A black hole is
%an object, demarcated by
a region of space-time
where the gravitational field is so intense
that neither matter nor radiation can escape.
There is a natural ``gravitational radius'' associated with a mass $m$,
called the Schwarzschild radius, given by
\begin{equation}
  r_{\rm Schwarz}(m) = \frac{2 G m}{c^2} = 2.95 \, \biggl(
  \frac{m}{\Msun} \biggr) \, {\rm km},
\end{equation}
where
$\Msun =1.99 \times 10^{30}~\rm{kg}$
is the mass of the Sun,
$G = 6.67 \times 10^{-11} \; {\rm m^3/s^2kg}$
is Newton's gravitational constant, and
$c = 2.998 \times 10^8 \; {\rm m/s}$
is the speed of light.
According to the hoop conjecture, if a non-spinning mass is compressed
to within that radius, then it must form a black hole~\cite{Thorne:1972ji}.
Once the \BH is formed,
any object that comes within this radius can no longer
escape out of it.

Here, the result that GW150914 was emitted by an inspiral and merger of two \BHs
follows from
(1) the strain data visible at the instrument output,
(2) dimensional and scaling arguments,
(3) primarily Newtonian orbital dynamics and
(4) the Einstein quadrupole formula for the luminosity of a \GW source\footnote{
In the terminology of GR corrections to Newtonian dynamics,
(3) \& (4) constitute the ``0th post-Newtonian" approximation (0PN)
(see \refsec{Sec:just:newton}).
A similar approximation was used for the first analysis of
binary pulsar PSR 1913+16 \cite{Hulse:1974eb, Taylor:1982zz}.
}.
These calculations are simple enough that they can be readily
verified with pencil and paper in a short time.
Our presentation is by design approximate, emphasizing simple arguments.

Specifically, while the orbital motion of two bodies is approximated
by Newtonian dynamics and Kepler's laws
to high precision at sufficiently large separations and sufficiently low velocities,
we will invoke Newtonian dynamics
to describe the motion even toward the end point of orbital motion
(We revisit this assumption in \refsec{Sec:just:newton}).
The theory of \GR
is a fully nonlinear theory,
which could make
any Newtonian analysis
wholly unreliable;
however, solutions of Einstein's equations
using numerical relativity
(NR)\cite{Pretorius:2005gq, Buonanno:2006ui, Boyle:2007ft}
have shown that
a binary system's departures
from Newtonian dynamics
can be described well using
a quantifiable analytic perturbation
until quite late in its evolution
- late enough for our argument
(as shown in \refsec{Sec:just:newton}).

The approach presented here,
using basic physics,
is intended
as a pedagogical introduction to the physics of \GW signals,
and as a tool to build intuition using rough, but straightforward, checks.
Our presentation here is by design elementary,
but gives results consistent with more advanced treatments.
The fully rigorous arguments,
as well as precise numbers describing the system,
have already been published elsewhere\cite{PEPaper, BetterPE, TestingGRPaper, O1BBH, AstroPaper}.

The paper is organized as follows:
our presentation begins with the data output by the detectors\footnote{
The advanced LIGO detectors use laser interferometry to measure the strain
caused by passing \GWsns.
For details of how the detectors work,
see \inlinecite{DetectionPaper} and its references.}
.
\refsec{Sec:data} describes the properties of the signal read off the strain data,
and how they determine the quantities relevant for analyzing the system as a binary inspiral.
We then discuss in \refsec{Sec:basic}, using the simplest assumptions,
how the binary constituents must be heavy and small, consistent only with being black holes.
In \refsec{Sec:justifications}
we examine and justify the assumptions made,
and constrain both masses to be well above the heaviest known neutron stars.
\refsec{Sec:luminosity} uses the peak \GW luminosity
to estimate the distance to the source,
and calculates the total luminosity of the system.
The appendices provide a calculation of gravitational radiation strain and radiated power (\refapp{Sec:Mc from Quadrupole}),
and discuss astrophysical compact objects of high mass (\refapp{Sec:stability})
and what one might learn from the waveform after the peak (\refapp{Sec:ringdown}).

\begin{figure}[h!]
\centering
\includegraphics[width=\columnwidth]{figs/OmegaScanDetPaper1.jpg}
\caption{A representation of the strain-data as a time-frequency
		   plot (taken from \inlinecite{DetectionPaper}),
		   where the increase in signal frequency (``chirp")
		   can be traced over time.}
\label{f:time-freq-plot}
\end{figure}

\section{Analyzing the observed data}
\label{Sec:data}

Our starting point is shown in \refigure{f:rawdata}:
the instrumentally observed strain data $h(t)$,
after applying a band-pass filter to the LIGO sensitive frequency band (35--350 Hz),
and a band-reject filter around known instrumental noise frequencies \cite{LOSC}.
The time-frequency behavior of the signal is depicted in \refigure{f:time-freq-plot}.
An approximate version of the time-frequency evolution can also be obtained
directly from the strain data in \refigure{f:rawdata}
by measuring the
time differences $\Delta t$ between
successive zero-crossings
\footnote{
To resolve the crossing at
$t\sim0.35$ s,
when the signal amplitude is lower
and the true waveform's sign transitions
are difficult to pinpoint,
we averaged the positions of the
five adjacent zero-crossings
(over $\sim6$ ms).
}
and estimating
$\fgw=1/(2\Delta t)$,
without assuming a waveform model.
We plot the $-8/3$ power of these estimated frequencies in \refigure{f:linearfit},
and explain its physical relevance below.

The signal is dominated by several cycles of a wave pattern
whose amplitude is initially increasing,
starting from around the time mark 0.30 s.
In this region the gravitational wave period is decreasing, thus the frequency is increasing.
After a time around 0.42s, the amplitude drops rapidly,
and the frequency appears to stabilize.
The last clearly visible cycles
(in both detectors, after accounting for a $6.9$~ms time-of-flight-delay \cite{DetectionPaper})
indicate that the final instantaneous frequency is above $200$~Hz.
The entire visible part of the signal lasts for around $0.15~$s.

In general relativity,
\GWs are produced by accelerating masses\cite{Flanagan:2005yc}.
Since the waveform clearly shows at least eight oscillations,
we know that mass or masses are oscillating.
The increase in \GW frequency and amplitude also indicate that
during this time the oscillation frequency of the source system is increasing.
This initial phase cannot be due to a perturbed system returning back to stable equilibrium,
since oscillations around equilibrium are generically characterized by
roughly constant frequencies and decaying amplitudes.
For example, in the case of a fluid ball,
the oscillations would be damped by viscous forces.
Here, the data demonstrate very different behavior.

During the period when the \GW frequency and amplitude are increasing,
orbital motion of two bodies is the only plausible explanation:
there, the only ``damping forces'' are provided by \GW emission,
which brings the orbiting bodies closer (an ``inspiral"),
increasing the orbital frequency and
amplifying the \GW energy output from the system\footnote{
The possibility of a different inspiraling system,
whose evolution is not governed by \GWsns,
is explored in \refapp{app:different inspiral}
and shown to be inconsistent with this data.
}.
%

Gravitational radiation
has many aspects analogous to
electromagnetic (EM) radiation from accelerating charges.
A significant difference is that there is no analog to EM
dipole radiation, whose amplitude is proportional
to the second time derivative of the electric dipole moment.
This is because the gravitational analog
is the mass dipole moment
($\sum_A m_A {\bf x}_A$ at leading order in the velocity)
whose first time derivative
is the total linear momentum,
which is conserved for a closed system,
and whose second derivative therefore vanishes.
Hence, at leading order,
gravitational radiation is quadrupolar.
Because the quadrupole moment
(defined in \refapp{Sec:Mc from Quadrupole})
is symmetric under rotations by $\pi$
about the orbital axis,
the radiation has a frequency
{\em twice} that of the orbital frequency
(for a detailed calculation for a 2-body system,
see \refapp{Sec:Mc from Quadrupole}
and pp. 356-357 of \inlinecite{LandauLifshitz}).

The eight \GW cycles of increasing frequency therefore require
at least four orbital revolutions, at separations large enough
(compared to the size of the bodies) that the bodies do not collide.
The rising frequency signal eventually terminates,
suggesting the end of inspiraling orbital motion.
As the amplitude decreases and the frequency stabilizes
the system returns to a stable equilibrium configuration.
We shall show that the only reasonable explanation for the
observed frequency evolution is that the system consisted
of two black holes that had orbited each other and subsequently merged.

{\bf Determining the frequency at maximum strain amplitude $\fgwmax$}:
The single most important quantity for the reasoning in this paper is
the \GW frequency at which the waveform has maximum amplitude.
Using the zero-crossings around the peak of \refigure{f:rawdata}
and/or the brightest point of \refigure{f:time-freq-plot},
%\refigures{f:rawdata} and \ref{f:time-freq-plot},
we take the conservative (low) value
\begin{equation}
  \fgwmax \sim \CHIRPFSTRAINPEAK ~\rm{Hz},
\end{equation}
where here and elsewhere the notation indicates that the quantity
before the vertical line is evaluated at the time indicated after the line.
We thus interpret the observational data as indicating that the bodies were orbiting each other
(roughly Keplerian dynamics)
up to at least an orbital angular frequency
\begin{equation}
  \wkepmax = \frac{2\pi \fgwmax}{2} = 2\pi \times 75~\rm{Hz}.
\end{equation}

{\bf Determining the mass scale}:
Einstein found\cite{EinsteinGW} that the \GW strain $h$
at a (luminosity) distance $d_L$ from a system
whose traceless mass quadrupole moment is $Q_{ij}$
(defined in \refapp{Sec:Mc from Quadrupole})
is
\begin{equation}
  \label{e:quadrupole strain}
  h_{ij} = \frac{2 \, G}{c^4 \, d_L} \frac{\D^2 Q_{ij}}{\D t^2},
\end{equation}
and that the rate at which energy is carried away by these \GWs
is given by the quadrupole formula\cite{EinsteinGW}

\begin{eqnarray}
  \label{e:quadrupole2}
  \frac{\D E_{\rm GW}}{\D t}
  & = & \frac{c^3}{16\pi \, G} \iint \bigl| \dot{h} \bigr|^2
      \D S  
      =  \frac{1}{5} \frac{G}{c^5}\sum_{i,j=1}^3  \frac{\D^3 Q_{ij}}{\D t^3}\frac{\D^3 Q_{ij}}{\D t^3}  ~,\\
\nonumber
&& {\rm where\ } \bigl| \dot{h} \bigr|^2 =  \sum_{i,j=1}^3
      \frac{\D h_{ij}}{\D t}
      \frac{\D h_{ij}}{\D t} ,
\end{eqnarray}
the integral is over a sphere at radius $d_L$
(contributing a factor $4\pi d_L^2$),
and the quantity on the right-hand side must be averaged
over (say) one orbit\footnote{See
\refapp{Sec:Mc from Quadrupole}
for a worked-out calculation,
and pp. 974-977 of \inlinecite{MTW}
for a derivation of these results,
obtained by linearizing the Einstein Equation,
the central equation of general relativity.}.

%Taking a time derivative of \refeqn{e:quadrupole strain},
%plugging in \refeqn{e:quadrupole1}

In our case, \refeqn{e:quadrupole2}
gives the rate of loss of orbital energy to \GWsns,
when the velocities of the orbiting objects are not too close
to the speed of light,
and the strain is not too large\cite{Flanagan:2005yc};
we will apply it until the frequency $\fgwmax$,
see \refsec{Sec:just:newton}.
This wave description is applicable
in the ``wave zone" \cite{wavezone},
where the gravitational field is weak
and the expansion of the universe is ignored
(see \refsec{Sec:just:distance}).

For the binary system we denote the two masses by $m_1$ and $m_2$,
the total mass by $M=m_1 + m_2$,
and the reduced mass by $\mu = m_1 m_2 / M$.
We define the mass ratio $q=m_1/m_2$ and
without loss of generality assume that $m_1 \ge m_2$ so that $q \ge 1$.
To describe the \GW emission from a binary system, a useful mass quantity
is the {\it chirp mass}, $\Mc$, related to the component masses by
\begin{equation}
  \Mc= \frac{(m_1 m_2)^{3/5}}{(m_1 + m_2)^{1/5}} ~.
\label{e:Mc1}
\end{equation}

Using Newton's laws of motion, Newton's universal law of gravitation, and
Einstein's quadrupole formula for the \GW luminosity of a system, a
simple formula is derived in \refapp{Sec:Mc from Quadrupole}
(following \inlinecites{Peters:1963ux, Peters:1964qza})
relating the frequency and frequency derivative
of emitted \GWs to the chirp mass,
\begin{equation}
\label{e:Mc2}
% \dot f = \frac{96}{5}\pi^{8/3} \left(\frac{G\,\Mc}{c^3}\right)^{5/3} f^{11/3},
\Mc = \frac{c^3}{G} \( \(\frac{5}{96}\)^{3} \pi^{-8} \(f_{\rm GW}\)^{-11} \(\dot{f}_{\rm GW}\)^{3} \)^{1/5} ,
\end{equation}
where $\dot{f}_{\rm GW} = \D f_{\rm GW}/\D t$ is the rate-of-change of the frequency
(see \refeqn{e:Mc derivation} and Eq.~3 of
\inlinecite{PhysRevLett.74.3515}).
This equation is expected to hold as long as the Newtonian approximation is valid
(see \refsec{Sec:just:newton}).

Thus, a value for the chirp mass can be determined
directly from the observational data,
using the frequency and frequency derivative
of the \GWs at any moment in time.
For example, values for the frequency can be estimated
from the time-frequency plot of the observed \GW strain data
(\refigure{f:time-freq-plot}),
and for the frequency derivative
by drawing tangents to the same curve.
The time interval during which
the inspiral signal is in
the sensitive band of the detector
(and hence is visible) corresponds to
\GW frequencies in the range
$30<\fgw<\CHIRPFSTRAINPEAK$~Hz.
Over this time,
the frequency (period) varies
by a factor of 5 ($\frac{1}{5}$),
and the frequency derivative varies
by more than two orders-of-magnitude.
The implied chirp mass value, however,
remains constant to within $35\%$.
The exact value of $\Mc$ is not critical
to the arguments that we present here,
so for simplicity we take
$\Mc = \CHIRPMASSAPPROX\!$.

{\bf Note that the characteristic mass scale of the radiating system
  is obtained by direct inspection of the time-frequency behavior of
  the observational data.}

The fact that the chirp mass remains approximately constant
for $\fgw \!<\! \CHIRPFSTRAINPEAK~$Hz
is strong support for the orbital interpretation.
The fact that the amplitude of the \GW strain
increases with frequency also supports this interpretation,
and suggests that the assumptions that go into the
calculation which leads to these formulas are applicable:
the velocities in the binary system are not too close to the speed of light,
and the orbital motion has an adiabatically changing radius and period
described instantaneously by Kepler's laws.  The data also indicate
that these assumptions certainly break down at a \GW frequency above $\fgwmax$,
as the amplitude stops growing.

Alternatively, \refeqn{e:Mc2} can be integrated to obtain
\begin{equation}
f_{\rm GW}^{-8/3} (t) = \frac{(8\pi)^{8/3}}{5} \( \frac{G\, \Mc}{c^3} \)^{5/3} \, (t_c - t),
\label{e:feightthirds}
\end{equation}
which does not involve $\dot{f}_{\rm GW}$ explicitly,
and can therefore be used to calculate $\Mc$ directly
from the time periods between zero-crossings in the strain data.
We have performed such an analysis,
presented in \refigure{f:linearfit},
to find similar results.
We henceforth adopt a conservative lower estimate of $\CHIRPMASSAPPROX$.
We remark that this mass is derived from quantities measured in the detector frame,
thus it and the quantities we derive from it are given in the detector frame.
Discussion of redshift from the source frame appears in \refsec{Sec:just:distance}.

\begin{figure}
\includegraphics
[width=\columnwidth]
{figs/linearfitV2Opt7.png}
\caption{A linear fit (green) of $\fgw^{-8/3}(t)$.
  While this interpolation used
  the combined strain data from H1 and L1
  (in fact, the sum of L1
  with time shifted and sign-flipped H1,
  as explained),
  A similar fit can be done using
  either H1 or L1 strain independently.
  The fit shown has residual sum of squares
  $R_{\rm L1-H1}^2\sim0.9$;
  we have also found
  $R_{\rm H1}^2\sim0.9$
  and $R_{\rm L1}^2\sim0.8$.
  The slope of this fitted line gives
  an estimate of the chirp mass
  using \refeqn{e:feightthirds}.
  The blue and red lines indicate
  $\Mc$ of 30\Msun and 40\Msun, respectively.
  The error-bars have been estimated
  by repeating the procedure for
  waves of the same amplitudes and frequencies
  added to the LIGO strain data just before GW150914.
  A similar error estimate has been found using
  the differences between H1 and L1 zero-crossings.
}
\label{f:linearfit}
\end{figure}

\begin{figure}
\includegraphics[width=0.7\columnwidth]{figs/keplerorbit.png}
\centering
\caption{A demonstration of the scale of the orbit at minimal
  separation (black, 350 km) vs. the scale of the compact radii:
  Schwarzschild (red, diameter 200 km)
  and extremal Kerr (blue, diameter 100 km).
%  The figure is drawn to scale.
  Note the masses here are equal;
  as \refsec{Sec:just:q} explains,
  the system is even more compact for unequal masses.
%
  While identification of a rigid reference frame for
  measuring distances between points is not unique in relativity,
  this complication only really arises with strong gravitational fields,
  while in the Keplerian regime
  (of low compactness and low gravitational potentials)
  the system's center-of-mass rest-frame can be used.
  Therefore if the system is claimed to be non-compact,
  the Keplerian argument should hold,
  and constrain the distances to be compact.
  Thus the possibility of non-compactness
  is inconsistent with the data;
  see also \refsec{Sec:just:newton}.
}
\label{fig:tightorbit}
\end{figure}

\section{Evidence for compactness in the simplest case}
\label{Sec:basic}

For simplicity, suppose that the two bodies have equal masses, $m_1 =
m_2$. The value of the chirp mass then implies that $m_1 = m_2 =
2^{1/5} \Mc = \MCOMPONENTAPPROX$, so that the total mass would be $M=m_1 + m_2 =
\MFINALAPPROX$.
We also assume for now that the objects are not spinning, and that their orbits remain
Keplerian and essentially circular until the point of peak amplitude.

Around the time of peak amplitude the bodies therefore had an orbital separation $R$ given by
\begin{equation}
  R = \( \frac{G M}{\wkep^2\submax} \)^{1/3} = \ORBITALSEPAPPROX~\rm{km}.
  \label{e:keplerradius}
\end{equation}
%
Compared to normal length scales for stars, this is a {\it tiny} value.
This constrains the objects to be exceedingly small,
or else they would have collided and merged
long before reaching such close proximity.
Main-sequence stars have radii measured in hundreds of thousands or millions of kilometers,
and white dwarf (WD) stars have radii which are typically ten thousand kilometers.
Scaling \refeqn{e:keplerradius} shows that such stars' inspiral evolution
would have terminated with a collision at an orbital frequency of a few mHz (far below 1~Hz).

The most compact stars known are neutron stars, which have radii of
about ten kilometers.  Two neutron stars could have orbited at this
separation without colliding or merging together -- but the maximum
mass that a neutron star can have before collapsing into a black hole
is about 3~$\Msun\!\!$ (see \refapp{Sec:stability}).

In our case, the bodies of mass
$m_1 = m_2 = \MCOMPONENTAPPROX$ each have a
Schwarzschild radius of $103$~km.
This is illustrated in \refigure{fig:tightorbit}.
The orbital separation of these objects, \ORBITALSEPAPPROX~km,
is only about twice the sum of their Schwarzschild radii.

In order to quantify the closeness of the two
objects relative to their natural gravitational radius, we
introduce the compactness ratio $\mathcal{R}$.
This is defined as the
Newtonian orbital separation between the centers of the objects
divided by the sum of their smallest possible respective radii (as compact objects).
For the non-spinning, circular orbit, equal-mass case just discussed
$\mathcal{R} = {\ORBITALSEPAPPROX \, \rm km}/{206 \, \rm km} \sim 1.7$.

For comparison with other
known Keplerian systems,
the orbit of Mercury,
the innermost planet in our solar system,
has $\mathcal{R} \sim 2 \times 10^7$,
the binary orbit for
the stellar \BH in Cyg X-1\footnote{Radio,
optical and X-ray telescopes
have probed the accretion disk extending
much further inside\cite{CygX1}.
}
has $\mathcal{R} \sim 3\times10^5$,
and the binary system of highest known orbital frequency,
the WD system HM Cancri (RX J0806),
has $\mathcal{R} \sim 2 \times 10^4$
\cite{Postnov:2014tza}.
Observations of orbits around our galactic center
indicate the presence of a supermassive \BHns,
named Sgr A* \cite{Gillessen:2008qv, Ghez:2008ms},
%Its mass is about $4\times10^6 \Msun\!$,
%corresponding to a Schwarzschild radius of $\sim10^7\,{\rm km}$,
%while direct telescope observations
%at resolutions as fine as $\sim4\times10^7\,{\rm km}$,
%have not seen it
%confining it to $\sim\mathcal{R} \lesssim 4$.
with the star S2 orbiting it
as close as
$\mathcal{R} \sim 10^3$.
For a system of two \NSs just touching,
$\mathcal{R}$ would be between
$\sim\!2$ and $\sim\!5$.


The fact that the Newtonian/Kepler evolution of the
orbit inferred from the signal of GW150914
breaks down when the separation is about the order of the
\BH radii (compactness ratio $\mathcal{R}$ of order 1) is further
evidence that the objects are highly compact.

\section{Revisiting the assumptions}
\label{Sec:justifications}

In \refsec{Sec:basic} we used the data to show that the
coalescing objects are black holes under the assumptions of a circular orbit,
equal masses, and no spin.
It is not possible, working at the level of approximation that we are using here,
to directly constrain these parameters of the system
(although more advanced techniques
are able to constrain them, see \inlinecite{PEPaper}).
However, it is possible to examine how these assumptions affect our conclusions
and in this section we show that
relaxing them does not significantly change the outcome. 
We also use the Keplerian approximation
to discuss these three modifications (Sec. \ref{Sec:just:ecc}-\ref{Sec:just:spins}),
then revisit the Keplerian assumption itself,
and discuss the consequences of foregoing it
(Sec. \ref{Sec:just:newton}-\ref{Sec:just:chirpmass}).
In \refsec{Sec:just:distance} we discuss
the distance to the source,
and its potential effects.

\caveatformat{Orbital eccentricity}
\label{Sec:just:ecc}
For non-circular orbits
with eccentricity $e > 0$,
the $R$ of Kepler's third law
(\refeqn{e:keplerradius}) no longer refers to the
orbital separation but rather to the semi-major axis.
The instantaneous orbital separation $r_{\rm sep}$
is bounded from above by $R$,
and from below by the point of closest approach (periapsis),
$r_{\rm sep} \ge \(1-e\) R$.
We thus see that the compactness bound
imposed by eccentric orbits is even tighter
(the compactness ratio $\mathcal{R}$ is smaller).

There is also a correction to the luminosity
which depends on the eccentricity.
However, this correction is significant
only for highly eccentric orbits\footnote{
Eccentricity increases
the luminosity\cite{Peters:1963ux, Peters:1964qza}
by a factor
$\ell(e)=\(1-e^2\)^{-7/2} \(1 + \frac{73}{24}e^2 + \frac{37}{96}e^4 \) \geq 1$,
thus reducing the chirp mass (inferred using \refeqn{e:Mc2})
to $\Mc(e) = \ell^{-3/5}(e) \cdot \Mc(e\!=\!0)$.
Taking into account the ratio between
the separation at periapsis and the semi-major axis,
one obtains
$\mathcal{R}(e)=\(1-e\)\ell^{2/5}(e)\cdot\mathcal{R}(e\!=\!0)$.
Hence for the compactness ratio to increase,
the eccentricity must be $e \gtrsim 0.6$,
and for a factor of 2, $e \gtrsim 0.9$
(see \refigure{fig:separationAndFrequency})}.
For these,
the signal should display a modulation\cite{Tanay:2016zog}:
the velocity would be greater near periapsis
than near apoapsis,
so the signal would alternate between
high-amplitude and low-amplitude peaks.
Such modulation is not seen in the data,
whose amplitude grows monotonically.

This is not surprising,
as the angular momentum
that \GWs carry away causes the orbits to circularize
much faster than they shrink\cite{Peters:1963ux, Peters:1964qza}.
This correction can thus be neglected.

\caveatformat{The case of unequal masses}
\label{Sec:just:q}
It is easy to see that the compactness ratio $\mathcal{R}$ also gets smaller
with increasing mass-ratio,
as that implies a higher total mass for the observed value of the Newtonian order chirp mass.
To see this explicitly,
we express the component masses and total mass in terms of the chirp mass $\Mc$ and the mass ratio $q$, as
$m_1 = \Mc (1+q)^{1/5}q^{2/5}$, $m_2 =  \Mc (1+q)^{1/5}q^{-3/5}$, and
%
\begin{equation}
  M  =  m_1 + m_2 = \Mc (1 + q)^{6/5}q^{-3/5}.
  \label{m_as_mchirp_and_q}
\end{equation}
%
The compactness ratio $\mathcal{R}$ is the ratio of the orbital
separation $R$ to the sum of the Schwarzschild radii of the two
component masses, $r_{\rm Schwarz}(M) = r_{\rm Schwarz}(m_1) + r_{\rm
  Schwarz}(m_2)$, giving
%
\bea
\mathcal{R}&=&\frac{R}{r_{\rm Schwarz}(M)}
	= \frac{c^2}{2(\wkepmax GM)^{2/3}}	\nonumber \\
	&=& \frac{c^2}{2(\pi \fgwmax G\Mc)^{2/3}} \frac{q^{2/5}}{(1\!+\!q)^{4/5}}
	\approx \frac{3.0 \,q^{2/5}}{(1\!+\!q)^{4/5}}\,.
\label{eq:separation_massRatio}
\eea
%
This quantity is plotted in \refigure{fig:separationAndFrequency},
which clearly shows that for mass ratios $q>1$ the compactness ratio
{\it decreases}: the separation between the objects becomes smaller
when measured in units of the sum of their Schwarzschild radii.  Thus,
for a given chirp mass and orbital frequency, a system composed of
unequal masses is {\it more} compact than one composed of equal masses.

\begin{figure}[t]
\centering
\includegraphics[width=1\columnwidth]{figs/mass_ecc_blue2.png}
\caption{
  This figure shows the compactness ratio constraints
  imposed on the binary system by
  $\Mc = 30 \, \Msun$ and $\fgwmax = 150 \, {\rm Hz}$.
  It plots the compactness ratio (the ratio of the separation between
  the two objects to the sum of their Schwarzschild radii) as a function of
  mass ratio and eccentricity
  from $e=0$ to the very high (arbitrary) value of $e=0.8$.
  The bottom-left corner ($q=1,e=0$)
  corresponds to the case given in \refsec{Sec:basic}.
  At fixed mass ratio, the system becomes more compact
  with growing eccentricity until $e=0.27$,
  as explained in \refsec{Sec:just:ecc}.
  The bottom edge ($e=0$) illustrates the argument given in
  \refsec{Sec:just:q} and \refeqn{eq:separation_massRatio}:
  the system becomes more compact as the mass ratio increases.
  We note that (for $e=0$) beyond mass ratio of
  $q\sim 13$ ($m_2\sim11\,\Msun\!$) the system 
  would become more compact than the sum of
  the component Schwarzschild radii. 
}
\label{fig:separationAndFrequency}
\end{figure}

One can also place an {\it upper} limit on the mass ratio $q$,
thus a lower bound on the smaller mass $m_2$, based purely on the data. 
This bound arises from minimal compactness:
we see from the compactness ratio plot in \refigure{fig:separationAndFrequency} 
that beyond the mass ratio of $q\sim 13$ the system becomes so compact
that it will be within the Schwarzschild radii of the combined mass of the two bodies.
 This gives us a limit for the mass of the smaller object
$m_2 \geq 11 \, \Msun$.
As this is 3--4 times more massive than the \NS limit,
both bodies are expected to be \BHs.

\caveatformat{The effect of objects' spins}
\label{Sec:just:spins}
The third assumption we relax concerns the spins of the objects.
For a mass $m$ with spin angular momentum $S$ we define
the dimensionless spin parameter
\be
\chi = \frac{c}{G}\frac{S}{m^2}~.
\ee
The spins of $m_1$ and $m_2$ modify their gravitational radii 
as described in this subsection,
as well as the orbital dynamics, as described in the next subsection.

The smallest radius a non-spinning object ($\chi=0$)
could have without being a
\BH is its Schwarzschild radius.
Allowing the objects to have angular momentum (spin) pushes
the limit down by a factor of two, to the radius of an extremal Kerr
\BH (for which $\chi=1$), $r_{\rm EK}(m)=\frac{1}{2}r_{\rm Schwarz}(m)=Gm / c^2$.
As this is linear in the mass, and summing radii linearly,
we obtain a lower limit on the Newtonian separation of two adjacent non-\BH bodies of total mass $M$ is \be r_{\rm EK}
(m_1) + r_{\rm EK} (m_2)= \frac{1}{2} r_{\rm Schwarz}(M) = \frac{G \, M}{c^2}
\approx 1.5\(\frac{M}{\Msun}\)\,{\rm km}.
\label{rcompact}
\ee
The compactness ratio can also be defined in relation to $r_{\rm EK}$ rather than
$r_{\rm Schwarz}$, which is at most a factor of two larger than for non-spinning objects.

We may thus constrain the
orbital compactness ratio (now accounting for eccentricity, unequal masses, and spin) by
\bea \mathcal{R} \, &=& \,
\frac{r_{\rm sep}(M)}{r_{\rm EK}(M)} \, \leq \, \frac{R(M)}{r_{\rm EK}(M)} \, = \,
\frac{ c^2 } { \( G \, M \, \wkep \)^{2/3}} 	 \\
 &\leq& \, \frac{
  c^2 } { \( 2^{6/5} \, G \, \Mc \, \wkep \)^{2/3}} \, = \,
\frac{ c^2 } { \( 2^{6/5} \, \pi \, G \, \Mc \, \fgwmax \)^{2/3}} \,
\simeq \, 3.4 ~,	\nonumber
\label{compactness ratio}
\eea
where in the last step we used $\Mc=\CHIRPMASSAPPROX\!\!$ and $\fgwmax=\CHIRPFSTRAINPEAK$~Hz.
This constrains the constituents to under 3.4 (1.7) times their extremal Kerr (Schwarzschild) radii,
making them highly compact.
The compact arrangement is illustrated in Fig. \ref{fig:tightorbit}.
%Note that the changes in the Keplerian orbits that
%result from including relativistic effects in the gravitational potential
%would allow even tighter constraints on the orbital separation and compactness.

We can also derive an upper limit on the value of the mass ratio $q$,
from the constraint that the compactness ratio must be larger than
unity. This is because, for a fixed value of the chirp mass $\Mc$ and a fixed
value of $\fgwmax$,
the compactness ratio $\mathcal{R}$ decreases
as the mass ratio $q$ increases.
Thus, the constraint $\mathcal{R} \geq 1$,
puts a limit on the maximal possible $q$
and thus on the maximum total mass $M_{\rm max}$,
\be \( \frac{ M_{\rm max} } { \Mc } \) \, \simeq \, 3.4^{3/2} \times 2^{6/5}
\simeq 14.4 ~,
\label{M limit}
\ee which for GW150914 implies $M_{\rm max} \simeq 432\, \Msun\!\!$
(and $q \simeq 83$).
This again forces the smaller mass to be at least $5\, \Msun\!\!$
--
well above the \NS mass limit (\refapp{Sec:stability}).

The conclusion is the same as in the equal-mass or non-spinning case:
both objects must be black holes.

\caveatformat{Newtonian dynamics and compactness}
\label{Sec:just:newton}
We now examine the applicability of Newtonian dynamics.
The dynamics will depart from the Newtonian approximation
when the relative velocity $v$ approaches the speed of light
or when the gravitational energy becomes large
compared to the rest mass energy.
For a binary system
bound by gravity
and with orbital velocity $v$,
these two limits coincide
and may be quantified
by the post-Newtonian (PN) parameter \cite{lrr-2014-2}
$x = \left( v/c \right)^2\ = G\,M / \( c^2 \, r_{\rm sep}\)$.
Corrections to Newtonian dynamics may be
expanded in powers of $x$,
and are enumerated by their PN order.
The 0PN approximation is
precisely correct at $x=0$,
where dynamics are Newtonian and \GW emission
is described exactly by the quadrupole formula (\refeqn{e:quadrupole2}).

The expression for the dimensionless PN parameter
includes the Schwarzschild radius, so
$x$ can be immediately recast
in terms of the compactness ratio,
$x \sim \(2\mathcal{R}\)^{-1}$.
As Newtonian dynamics holds when $x$ is small, the Newtonian
approximation is valid down to compactness $\mathcal{R}$ of order of a few.
Arguing by contradiction,
if one assumes that the orbit is non-compact,
then our analysis of the data
using Newtonian mechanics
is justified as an approximation of \GR
and leads to the conclusion that the orbit is compact.

If either of the bodies is rapidly spinning,
their rotational velocity may also approach the speed of light,
modifying the Newtonian dynamics,
effectively adding
spin-orbit and spin-spin interactions.
However, these are also suppressed
with a power of the PN parameter
(1.5PN and 2PN, respectively \cite{lrr-2014-2,SpinEFT,SpinEFT2}),
and thus are significant only for compact orbits.

The same reasoning may also be applied
to the use of the quadrupole formula\cite{Flanagan:2005yc}
and/or to using the coordinate $R$ for the comparison of
the Keplerian separation to the corresponding compact object radii
(see \refigure{fig:tightorbit} and its caption),
as both of these are not entirely general and might be inaccurate.
The separations are also subject to some arbitrariness due to gauge freedom.
However here too, the errors in using these coordinates are non-negligible
only in the orbits very close to a \BHns,
so again this argument does not refute our conclusions.

\caveatformat{Is the chirp mass well measured? -- constraints on the individual masses}
\label{Sec:just:chirpmass}
As we are analyzing the final cycles before merger,
having accepted that the bodies were compact,
one might still ask whether
\refeqn{e:Mc2} correctly describes the chirp mass
in the non-Newtonian regime\cite{McWilliams:2010eq}.
In fact for the last orbits, it does not:
In Newtonian dynamics
stable circular orbits may exist
all the way down to merger,
and energy lost to \GWs drives
the inspiral between them.
However in general relativity,
close to the merger of compact objects
(at least when one of the objects is much larger than the other)
there are no such orbits past the
innermost stable circular orbit (ISCO),
whose typical location is given below.
Allowed interior trajectories must be non-circular
and ``plunge" inwards (see pp. 911 of \inlinecite{MTW}).
The changes in orbital separation and frequency
in the final revolutions
are thus not driven by the \GW emission
given by \refeqn{e:Mc2}.
This is why we used $\fgwmax$ at the peak,
rather than $\fgwfin$.

We shall now constrain the individual masses based on $\fgwfin$,
for which we do not need the Newtonian approximation at the late stage.
No \NSs have been observed above $3\,\Msun \!$;
we shall rely on an even more conservative \NS mass upper bound
at $4.76\,\Msun\!$, a value chosen because
given $\Mc$ from the early visible cycles, in order for
the smaller mass $m_2$ to be below this threshold, $m_1$ must be at
least $476\,\Msun\!$, which implies $q \geq 100$.
Is such a high $q$ possible with the data that we have?
Such a high mass ratio suggests a treatment of the system as
an extremal mass ratio inspiral (EMRI), where the smaller mass approximately
follows a geodesic orbit around the larger mass ($m_1\sim M$).
The frequencies of test-particle orbits (hence waveforms) around an object scale with
the inverse of its mass, and also involve its dimensionless spin $\chi$.
The orbital frequency $\w\orb$ as measured at infinity of a circular,
equatorial orbit at radius $r$ (in Boyer-Lindquist coordinates)
is given by \cite{Bardeen:1972fi}
%
\be
\label{eq:ModKep}
\w\orb = \frac{\sqrt{G M}}{r^{3/2}+\chi \, \( \sqrt{G M} / c\)^{3}}
%
%XXX THIS CAN ALSO BE WRITTEN AS (WHICH ONE LOOKS BETTER?)
%
= \frac{c^{3}}{G M}\left( \chi + \(\frac{c^2 r}{GM}\)^{3/2} \right)^{-1} %\nonumber
.
\ee
%

For example, around a Schwarzschild \BH ($\chi=0$) the quadrupole
\GW frequency at the \ISCO
(which is at $r = 6 GM/c^2$) is hence equal to
$f_{\rm GW} = 4.4 ( \Msun / M ) $~kHz,
while for an extremal Kerr \BH ($\chi=1$)
the orbital frequency at
ISCO ($r = GM/c^2$)
is $\w\orb = c^3/2GM$,
and the quadrupole gravitational frequency is 
$f_{\rm GW} = c^3/2\pi GM = 32 ( \Msun / M )$~kHz.
For a \GW from the final plunge,
the highest expected frequency is approximately
the frequency from the light ring (LR),
as nothing physical is expected
to orbit faster than light\footnote{Hypothesized
frequency up-conversions
due to nonlinear GR effects
have also been shown by NR
to be absent NR
\cite{Pretorius:2005gq, Buonanno:2006ui, Boyle:2007ft}.},
and as waves originating within the \LR
encounter an effective potential barrier
at the \LR going out\cite{
Davis:1971gg, Press:1971wr, Davis:1972ud, Buonanno:2000ef, Cardoso:2016rao}
.
The \LR is at
\begin{equation}
r_{\rm LR}= \frac{2\,G\,M}{c^2} \left(1 + \cos \left(\frac{2}{3}
\cos^{-1}(-\chi)\right)\right).
\label{eq:LR}
\end{equation}
This radius is $3GM/c^2$ for a Schwarzschild \BHns, while for a spinning Kerr \BHns,
as the spin $\chi$ increases the \LR radius decreases. For an extremal Kerr \BH it
coincides with the \ISCO at $GM/c^2$.
The maximal \GW frequency for a plunge into $m_1$ is then $67$~Hz.

Because we see \GW emission from orbital motion at frequencies much higher
than this maximal value, with or without spin, such a system is ruled out.
Hence even the lighter of the masses must be at least $4.76 ~\Msun\! > 3~\Msun\!$,
beyond the maximum observed mass of neutron stars.



\caveatformat{Possible redshift of the masses -- a constraint from the luminosity}
\label{Sec:just:distance}
Gravitational waves are stretched by the expansion of the Universe as they travel across it.
This increases the wavelength and decreases the frequency of the waves observed on Earth
compared to their values when emitted.  The same effect accounts for the redshifting of photons from distant objects.
The impact of this on the gravitational wave phasing corresponds to a scaling of the masses as measured on Earth;
dimensional analysis of \refeqn{e:Mc2} shows that
the source frame masses are smaller by $(1+z)$ relative to the detector frame,
where $z$ is the redshift.
Direct inspection of the detector data yields mass values from the red-shifted waves.
How do these differ from their values at the source?
In the next section, we estimate the distance to the source and hence the redshift, by
relating the amplitude and luminosity of the \GW from the merger
to the observed strain and flux  at the detector.
The redshift is found to be $z\le 0.1$, so the detector- and source-frame masses differ by less than 
of order 10$\%$.

\section{Luminosity and distance}
\label{Sec:luminosity}

Basic physics arguments also provide estimates of the peak
\GW luminosity of the system,
its distance from us,
and the total energy
radiated in \GWsns.

The \GW amplitude $h$ falls off
with increasing luminosity distance $d_L$
as $h \propto 1/d_L$.
As shown in \refigure{f:rawdata},
the measured strain peaks at $\hmax\sim10^{-21}$.
Had our detector been
ten times closer to the source,
the measured strain would have peaked
at a value ten times larger.
This could be continued,
but the scaling relationship would break
before $h$ reached unity,
because near the Schwarzschild radius
of the combined system $R\sim200~{\rm km}$
the non-linear nature of gravity would become apparent.
In this way we obtain a crude order-of-magnitude upper bound
\be
d_L < 10^{21} \times 200~{\rm km}
~ \sim ~ 6~{\rm Gpc}
\ee
on the distance to the source.

We can obtain a more accurate distance estimate
based on the luminosity, because 
the \GW luminosity from an equal-mass
binary inspiral has a peak value
which is independent of the mass.
This can be seen from 
naive dimensional analysis of the quadrupole formula,
which  gives a luminosity
$L \sim \frac{G}{c^5} M^2 r^4 \w^6$,
with $\w\sim c/r$ and $r\sim G M / c^2$,
and $M \w\sim c^3 /G$ for the final tight orbit.
Together this gives the Planck luminosity
\footnote{The ``Planck luminosity'' $c^5/G$ has been proposed as the
  upper limit on the luminosity
  of any physical system\cite{Dyson,
    Massa, Hogan:1999hz}.  Gibbons \cite{Gibbons} has suggested that
  $c^5/4G$ be called the ``Dyson luminosity'' in honor of the
  physicist Freeman Dyson and because it is a {\it classical} quantity
  that does not contain $\hbar$.},
\be
L\sim \Lplanck = c^5 / G = 3.6 \times 10^{52} \; {\rm W}~.
\ee
However, a closer look (\refeqn{e:quadrupole3}) shows
the prefactor should be $\frac{32}{5}\(\frac{\mu}{M}\)^2$,
which gives $\frac{2}{5}$ for an equal-mass system,
and is close to that for $q\sim1$.
Also, analysis of a small object falling into a Schwarzschild \BH
suggests $M\sim \frac{1}{6} c^2 r_{\rm ISCO} / G$ and $\w r \sim 0.5 c$.
Taken together with the correct exponents,
$L$ acquires a factor $0.4 \times 6^{-2} \times 0.5^6 \sim 0.2 \times 10^{-3}$.
While the numerical value may change by a factor of a few
with the specific spins,
we can treat its order of magnitude as universal
for similar-mass binaries.

Using \refeqn{e:quadrupole2} we relate the luminosity of \GWs to their strain $h$ at luminosity distance $d_L$,
\be
L \sim \frac{c^3 \, d_L^2}{4\,G} \bigl| \dot{h} \bigr|^2 \;  \sim \frac{c^5}{4\,G} \( \frac{\wgw d_L h}{c} \)^2~.
\ee
Thus we have
\be
\frac{L_{\rm peak}}{\Lplanck} \equiv \frac{L\submax}{\Lplanck} \sim0.2 \times 10^{-3} \sim \( \frac{\wgw d_L \hmax}{c} \)^2,
\ee
and we estimate the distance
from the change of the measured strain in time
over the cycle at peak amplitude,
as
\be
d_L \sim 45 ~ {\rm Gpc} \( \frac{\rm Hz}{\fgwmax}\) \(\frac{10^{-21}}{\hmax}\) ~,
\ee
which for GW150914 gives $d_L \sim 300 ~ {\rm Mpc}$.
This distance corresponds to a redshift of $z\le 0.1$,
and so does not substantially affect any of the conclusions.
For a different distance-luminosity calculation based only on the strain data
(reaching a similar estimate),
see \inlinecite{Burko:2016vnu}.

Using the orbital energy $E\orb$ 
(as defined in \refapp{Sec:Mc from Quadrupole})
we may also estimate the total energy
radiated as \GWs during the system's evolution
from a very large initial separation (where $E\orb^{\rm i}\to 0$)
down to a separation $r$.
For GW150914, using
$m_1\sim m_2 \sim \MCOMPONENTAPPROX$
and $r\sim R = \ORBITALSEPAPPROX \, {\rm km}$ (\refeqn{e:keplerradius}),
\begin{equation}
  \label{e:total energy}
  E_{\rm GW} = E\orb^{\rm i} - E\orb^{\rm f}
  = 0 - \( - \frac{G M \mu}{2 R} \)
  \sim 3 \,\Msun c^2.
\end{equation}
This quantity should be considered an estimate
for a lower bound on the total emitted energy
(as some energy is emitted in the merger and ringdown);
compare with the exact calculations in \inlinecite{DetectionPaper, PEPaper, BetterPE}.

We note that
the amount of energy emitted in this event
is remarkable.
During its ten-billion-year lifetime,
our sun is expected to convert
less than 1\% of its mass into light and radiation.
Not only did GW150914 release
$\sim300$ times as much energy
in \GWs
(almost entirely over
the fraction of a second
shown in \refigure{f:rawdata}),
but for the cycle at peak luminosity,
its power $L_{\rm peak}$ in the form of \GWs
was about 22 orders of magnitude greater
than the power output from our sun.

\section{Conclusions}
A lot of insight can be obtained by applying these basic physics arguments
to the observed strain data of GW150914.
These show the system that produced the \GW
was a pair of inspiraling black holes that approached very closely before merging.
The system is seen to settle down, most likely to a single
black hole.
Simple arguments can also give us information
about the system's distance and basic properties
(for a related phenomenological approach
see \inlinecite{Rodriguez:2016obk}).

With these basic arguments
we have only drawn limited conclusions
about the mass ratio $q$,
because the frequency evolution
described by \refeqn{e:Mc2}
does not depend on $q$.
The mass ratio $q$ does appear
in the PN corrections \cite{EIH1938, PhysRevLett.74.3515},
thus its value can be further constrained from the data
\cite{PEPaper, BetterPE}.

These arguments will not work for every signal,
for instance if the masses are too low to safely rule out a \NS constituent
as done in \refsec{Sec:just:chirpmass},
but should be useful for systems similar to GW150914.
There has already been another \GW detection, GW151226\cite{GW151226, O1BBH},
whose amplitude is smaller and therefore cannot be seen in the strain data
without application of more advanced techniques.

Such techniques,
combining analytic and numerical methods,
can give us even more information,
and we encourage the reader to explore how
such analyses and models have been used for
estimating the parameters of the system \cite{PEPaper, BetterPE},
for testing and constraining the validity of \GR
in the highly relativistic, dynamic regime \cite{TestingGRPaper}
and for astrophysical studies based on this event \cite{AstroPaper}.

We hope that this paper will serve as an invitation to the field,
at the beginning of the era of gravitational wave observations.

\begin{acknowledgements}
\input{LVCacknowledgments.txt}
\end{acknowledgements}

\appendix

\section{Calculation of gravitational radiation from a binary system}
\label{Sec:Mc from Quadrupole}
%%%
%%% Derivation of \Mc formula
%%%
Here we outline the calculation of the energy a binary system emits in \GWs
and the emitted energy's effect on the system.

\begin{figure}[t]
\centering
\includegraphics[width=\columnwidth]{figs/2BodyDiagram.pdf}
\caption{A two-body system, $m_1$ and $m_2$ orbiting
in the $xy$-plane
around their C.O.M.
}
\label{fig:two_bodies_diagram}
\end{figure}


First we calculate the quadrupole moment $Q_{ij}$ of the system's
mass distribution.
We use a Cartesian coordinate system ${\bf x}=(x_1, x_2, x_3)=(x,y,z)$
whose origin is the center-of-mass, with $r$ the radial distance from the origin. 
$\delta_{ij} = \rm{diag} (1,1,1)$ is the Kronecker-delta and
$\rho({\bf x})$ denotes the mass density.
Then
\bea
%\begin{equation}
Q_{ij} &=& \int \D^3 x  \, \rho({\bf x})  \bigl( x_i x_j - \frac{1}{3} r^2 \delta_{ij}
\bigr) \\
&=& \sum_{A\in\{1,2\}} \!\! m_A
\(
\begin{array}{ccc}
\frac{2}{3}x^2_A - \frac{1}{3}y^2_A 	&	 x_A y_A 										&	 0 \\
x_A y_A 										&	\frac{2}{3}y^2_A - \frac{1}{3}x^2_A  	&	 0 \\
0 												&	0  												&	 -\frac{1}{3}r^2_A
\end{array}
\)
,
\eea
%\end{equation}
where the second equality holds for a system of two bodies $A \in \{1,2\}$ in the $xy$-plane.
In the simple case of a circular orbit at separation $r=r_1+r_2$ and frequency $f=\frac{\w}{2\pi}$,
a little trigonometry gives for each object (see \refigure{fig:two_bodies_diagram})
\be
Q^A_{ij} (t) = \frac{m_A r_A^2}{2} I_{ij},
\ee
where
$I_{xx}=\cos(2\w t)+\frac{1}{3}$,
$I_{yy}=\frac{1}{3}-\cos(2\w t)$,
$I_{xy}=I_{yx}=\sin(2\w t)$
and $I_{zz} = -\frac{2}{3}$.
Combining we find $Q_{ij} (t) = \frac{1}{2} \mu r^2 I_{ij}$, 
%where we have used the standard reduced mass $\mu = m_1 m_2 / M$,
and the \GW luminosity from \refeqn{e:quadrupole2} is
\begin{equation}
  \label{e:quadrupole3}
  \frac{\D}{\D t} E_{\rm GW}
  = \frac{32}{5} \frac{G}{c^5} \mu^2 r^4 \w^6.
\end{equation}
This energy loss drains the orbital energy
$E\orb = - \frac{G M \mu}{2r}$,
thus
$\frac{\D }{\D t} E\orb
= \frac{G M \mu}{2r^2} \dot{r}
= - \frac{\D}{\D t} E_{\rm GW}$.
We assume that
the energy radiated away over each orbit
is small compared to $E\orb$,
in order to describe each orbit as approximately Keplerian.

Now using Kepler's third law $r^3 = GM / \w^2$
and its derivative $\dot{r}=-\frac{2}{3} r\dot{\w}/\w$
we can substitute for all the $r$'s and obtain
\begin{equation}
  \label{e:Mc derivation}
  \dot{\w}^3 = \(\frac{96}{5}\)^3 \frac{\w^{11}}{c^{15}} G^5 \mu^3 M^2 = \(\frac{96}{5}\)^3 \frac{\w^{11}}{c^{15}} \(G \Mc\)^5,
\end{equation}
having defined the chirp mass $\Mc = \( \mu^3 M^2 \)^{1/5} $.

We can see that \refeqn{e:Mc derivation}
describes the evolution of the system as an inspiral:
the orbital frequency goes up (``chirps''),
while by Kepler's Law the orbital separation shrinks.

\subsection{Gravitational radiation from a different rotating system}
\label{app:different inspiral}
A rising \GW amplitude can accompany a rise in frequency
in other rotating systems,
evolving under different mechanisms.
An increase in frequency means the system rotates faster and faster,
so unless it gains angular momentum, 
the system's characteristic length $r(t)$ should be decreasing.
For a system not driven by the loss of
energy and angular momentum to \GWsns,
rapidly losing angular momentum is also difficult,
thus the system should conserve its angular momentum
$L=\alpha M r^2 \w$,
and so $\w \propto L/r^2$.

The quadrupole formula (\refeqn{e:quadrupole strain})
then indicates the \GW strain amplitude should follow
the second time derivative of the quadrupole moment,
$h \propto M\, r^2 \, w^2 \propto L\,\w$.

Thus we see that for a system
\emph{not} driven by emission of \GWsns,
as the characteristic system size $r$ shrinks,
both its \GW frequency and amplitude grow,
but remain proportional to each other.
This is inconsistent with the data of GW150914
(\refigures{f:rawdata}, \ref{f:time-freq-plot}),
which show the amplitude only grows by a factor of about 2
while the frequency $\w(t)$ grows by at least a factor of 5.

\section{Possibilities for massive, compact objects}
\label{Sec:stability}
We are considering astrophysical objects with mass scale
$m \sim \MCOMPONENTAPPROX$, which are constrained to
fit into a radius $R $ such that the compactness ratio
obeys $\mathcal{R} = \frac{c^2 \, R}{G\,m}\lesssim 3.4$.
This produces a scale for their Newtonian density,
\begin{equation}
\label{density}
\rho \geq \frac{m}{(4\pi/3) R^3 } = 3 \times 10^{15} \; 
\left (\frac{3.4}{\mathcal{R}}\right )^3 
\left ( \frac{\MCOMPONENTAPPROX}{m} \right )^2 \; \frac{\rm kg}{{\, \rm m}^3} ~,
\end{equation}
where equality is attained for a uniform object.
This is a factor of $10^6$ more dense than white dwarfs, so we
can rule out objects supported by electron degeneracy pressure,
as well as any main-sequence star, which are less dense.
While this density is a factor of $\sim10^2$ {\it less} dense than neutron stars,
these bodies exceed the maximum \NS mass by
an order of magnitude,
as the \NS limit is
$\sim3\Msun\!\!$ (3.2~$\Msun\!\!$ in \inlinecites{Rhoades:1974fn,Hannam:2013uu},
2.9~$\Msun\!\!$ in \inlinecite{Kalogera:1996ci}).
A more careful analysis of the frequency change, including tidal
distortions, would have undoubtedly required the bodies to be
even more compact in order to reach the final orbital frequency.
This would push these massive bodies even closer to neutron-star
density, thus constraining  the equation of state into an even
narrower corner.
Thus, although theoretically a compactness ratio as low as $\mathcal{R}=4/3$
is permitted for uniform objects \cite{doi:10.1146/annurev-nucl-102711-095018},
we can conclude that the data do show that if any of these objects were material bodies,
they would need to occupy an extreme, narrow and heretofore unexplored and
unobserved niche in the stellar continuum.
The likeliest objects with such mass and compactness are black holes.


\section{Post inspiral phase: what we can conclude about the ringdown and the final object?}
\label{Sec:ringdown}

We have argued, using basic physics and scaling arguments, that the
directly observable properties of the signal waveform for \GW
frequencies $\fgw< \CHIRPFSTRAINPEAK$~Hz shows that the source had been
two black holes, which approached so closely that they subsequently merged.
We now discuss the properties of the signal waveform at higher frequencies,
and argue that this also lends support to this interpretation.

The data in Figures \ref{f:rawdata} and \ref{f:time-freq-plot} show that
after the peak \GW amplitude is reached,
the signal makes a few more complete cycles,
and continues to rise in frequency until reaching
about $\CHIRPFMAX$~Hz,
while dropping sharply in amplitude.
The frequency seems to level off just as the signal amplitude becomes
hard to distinguish clearly.

Is this consistent with a merger remnant \BHns?
Immediately after being formed in a merger, a black hole horizon is
very distorted.
It proceeds to ``lose its hair" and settle down to a final state of a Kerr \BHns, uniquely
defined\cite{NoHair} by its mass $M$ and spin parameter $\chi$.
Late in this ringdown stage, the remaining perturbations should
linearize, and the emitted \GW should thus have characteristic quasi-normal-modes
(QNMs). The set of QNMs is enumerated by various discrete indices, and their
frequencies and damping times are determined by $M$ and $\chi$. Each such set
would have a leading (least-damped) mode --
and so finding a ringdown of several cycles
with a fixed frequency would be strong evidence that a single final remnant was formed.
We do clearly see the \GW stabilizing in frequency (at around $\CHIRPFMAX$~Hz)
about two cycles after the peak, and dying out in amplitude.
Does the end of the observed waveform contain evidence of
an exponentially-damped sinusoid of fixed frequency?
Were such a mode found, analyzing its frequency and damping time,
in conjunction with a model for \BH perturbations,
could give an independent estimate of the mass and spin\cite{Press:1973zz}.

\caveatformat{Mode Analysis}
The ringing of a Kerr \BH can be thought of
as related to a distortion of space-time traveling on a \LR orbit outside
the black hole horizon (See \inlinecite{BertiCardosoWill} and references therein,
and Eqs.\ (\ref{eq:ModKep}, \ref{eq:LR}));
the expected frequency for a quadrupolar mode ($\ell=m=2)$ will thus be
given as a dimensionless complex number
\begin{equation}
\frac{G}{c^3} M \, \wgw = x + \i y.
\end{equation}
where the real part of $\wgw$
is the angular frequency and the imaginary part is the (inverse) decay time.
The ringdown amplitude and damping times are then found from
\begin{equation}
\e^{\i \wgw t } = \e^{\i\frac{c^3x}{GM}t} \, \e^{-\frac{c^3y}{GM}t} 
= \e^{2\pi \i \fRD t}  \, \e^{-t/\tau_{\rm damp}} \; ,
\end{equation}
to be
$\fRD  = c^3x/ \( 2 \pi GM \)$ and $\tau_{\rm damp}= GM/c^3y$.

The exact values of $x$ and $y$ can be found as when analyzing the normal
modes of a resonant cavity: one uses separation of variables to solve the field
equations, and then enforces the boundary conditions to obtain
a discrete set of complex eigenfrequencies \cite{BertiCardosoWill}.
However, limiting values on $x$, $x \in (\sim0.3,1]$, are derived immediately
from Eqs.\ (\ref{eq:ModKep}, \ref{eq:LR}),
with a factor of 2 between orbital and \GW frequencies.
The final \GW frequency is thus determined by the mass
(up to the order-of-unity factor $x$, which embodies the spin).
We have in fact already used this to show
how our high attained frequency constrains the total mass
and the compactness of the objects
(objects of larger radius would have distortion bulges orbiting much farther than the \LR,
mandating much lower frequencies).
For the parameter $y$ determining the damping time, numerical tabulations
of the QNMs \cite{BertiCardosoWill} show that
\begin{equation}
\fRD \; \tau_{\rm damp} = \frac{x}{2 \pi y } \sim 1 
\end{equation}
for a broad range of mode numbers and spins,
as long as the spin is not close to extremal.
This shows that the ringdown is expected to have a damping time
roughly equal to the period of oscillation. This is exactly what is
seen in the waveform, and is the reason the signal amplitude drops so low
by the time the remnant rings at the final frequency.

While it is beyond the scope of this paper to calculate the exact QNMs for \BHs of different spins,
or to find the final spin of a general \BH merger,
it is worth mentioning that for a wide range of spins for similar-mass binaries,
the final spin is expected to be about $\chi\sim0.7$,
for which Eq. (\ref{eq:ModKep}, \ref{eq:LR}) estimate that
${\rm Re} [ \frac{G}{c^3} M \wgw ] \sim 0.55$.

The exact value can be found using Table~II in \inlinecite{BertiCardosoWill},
where the leading harmonic ($\ell=2,m=2,n=0$)
for a \BH with a spin $\chi=0.7$ has
$\frac{G}{c^3}M\wgw = 0.5326 + 0.0808 i $,
giving a ringdown frequency
\begin{equation}
\fRD \approx 260~{\rm Hz}
\left ( \frac{65\,\Msun}{M} \right )  \; ,
\end{equation}
and a damping time
\begin{equation}
\tau_{\rm damp} =  4 \; {\rm ms} 
\left ( \frac{M}{65\,\Msun} \right )   \; \sim
\frac{1}{\fRD} .
\end{equation}
In other words, the signal in the data is fully consistent\cite{Burko:2016vnu} with
the final object being a Kerr black hole with a dimensionless spin
parameter $\chi \sim 0.7$ and a mass $M\sim 65\,\Msun{}$
Such a final mass is consistent with the merger of two \BHs
of $\sim\MCOMPONENTAPPROX$ each,
after accounting for the energy emitted as \GWs (\refeqn{e:total energy}).
This interpretation of the late part of the signal
is also consistent with numerical simulations \cite{PhysRevD.85.024018}.
Full numerical simulations from the peak and onward,
where the signal amplitude is considerably higher, also show
consistency with the formation of a Kerr \BH remnant \cite{TestingGRPaper, PEPaper}.


%%% The resulting bibliography-output (the contents of the .bbl file)
%%% must be pasted into this file before submission.
%%% 
\bibliographystyle{andp2012}
\bibliography{references}

%\begin{thebibliography}{00}
 
%\end{thebibliography}

\iftoggle{endauthorlist}{
%
% Restore the author, affiliation and maketitle commands.
%
\let\author\myauthor
\let\affiliation\myaffiliation
\let\maketitle\mymaketitle
%\title{Authors}
%\pacs{}
%\date{}
%\newpage
%\maketitle

\medskip
\centerline{\bf Authors}

\input{BBHBasicsADPAuthors.txt}
}


\end{document}
