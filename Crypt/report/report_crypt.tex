\documentclass[11pt]{article}
\usepackage[margin=2cm]{geometry}

\usepackage{multicol}

\title{DES Encryption Attack Report}
\author{wcrr51}
\date{January 2021}

\begin{document}
    \maketitle

    \begin{multicols*}{2}
        \section{Introduction}\label{sec:introduction}
        This report looks to explore an attack plan to decrypt the following 16-byte ciphertext:
        \begin{center}
            \texttt{0x903408ec4d951acfaeb47ca88390c475}
        \end{center}
        The following information is provided:
        \begin{itemize}
            \item The corresponding plaintext is a \textit{What Three Words} location
            \item The ciphertext was encrypted using DES in ECB mode with a 64-bit key
        \end{itemize}

        Firstly, as the \textit{What Three Words} format is $a.b.c$ where $a, b, c$ are words such that $n_a + n_b + n_c + 2 \leq 16$ where $n_x$ is the length (in characters) of word $x$.
        As the plaintext must contain two full stop characters (\texttt{0x2e}), these can be used to help check for a plaintext with brute-force decryption.

        While the key is advertised to be 64-bit, DES only uses 56 - the remaining 8 are either discarded or used as parity bits.
        This means the keyspace can be reduced by a factor of 256 by ignoring the last bit of each byte.


        \section{Chosen-plaintext Attack}\label{sec:chosen-plaintext-attack}


        \section{Attack Results}\label{sec:attack-results}
        After around 3 hours of carrying out the attack on an i7-6700k, the key \texttt{0x98a1bef23455dc03} is found, decrypting the provided ciphertext yields the 16-byte plaintext \texttt{tile.bills.print}.
        In \textit{What Three Words}, this refers to the coordinates \texttt{40.026102, -75.030026} in Philadelphia, Pennsylvania, USA\@.
    \end{multicols*}

\end{document}